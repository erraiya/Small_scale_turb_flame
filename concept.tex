Turbulence-flame interactions are often quantitatively investigated by means of characterizing
the two-way coupling between the turbulent flow field and the gradient of reactive scalar $\mathcal{Y}$.
%
This scalar can either be the mass fraction of a species present in the mixture, or the progress
variable which is a normalized quantity taking the values 0 and 1 in the fresh gases and the
burnt gases respectively.
%
The progress variable can be defined as the combination of a subset of species or in terms of the
temperature.
%
The evolution of the scalar $\mathcal{Y}$, which is considered as a progress variable in the following, 
is governed by the following transport equation
%
\begin{equation}
\frac{D\mathcal{Y}}{Dt} = \frac{1}{\rho} \nabla \cdot \left(\rho \mathcal{D}_{\mathcal{Y}}
\nabla \mathcal{Y}\right) +\dot{\omega}_{\mathcal{Y}}
\label{eq:trans_scalar}
\end{equation}
%
where $\mathcal{D}_{\mathcal{Y}}$ and $\dot{\omega}_{\mathcal{Y}}$ denote respectively the scalar's 
molecular diffusivity and the chemical reaction rate.
%
As the premixed flame evolves in the presence of turbulent fluctuations, the instantaneous flame 
front is subject to wrinkling which can be quantified using the variations and orientations of 
the flame normals pointing towards fresh gases and defined as :
\begin{equation}
\mathbf{n} = -\frac{\nabla \mathcal{Y}}{ \|\nabla \mathcal{Y}\|}
\end{equation}
%
where $\|\nabla \mathcal{Y}\| = \left[\mathcal{Y}^T\cdot\mathcal{Y}\right]^{1/2}$ is the scalar 
gradient magnitude, which can be used as a measure of the inverse of the local flame width $\delta_t$.
%
The dynamics of the scalar gradient magnitude, the local flame width and the normals to the 
iso-scalar surfaces are characterized with the equations

\begin{align}
\frac{D(\|\nabla \mathcal{Y}\|)}{Dt} &= - \mathbf{n} \cdot \frac{D(\nabla \mathcal{Y})}{Dt}
\label{eq:trans_magn_grad_c}\\
%
\frac{D(\|\nabla \mathcal{Y}\|^{-1})}{Dt} &= \|\nabla \mathcal{Y}\|^{-2 }\mathbf{n} \cdot 
\frac{D(\nabla \mathcal{Y})}{Dt}
\label{eq:trans_thickness}\\
%
\frac{D\mathbf{n}}{Dt} &= - \|\nabla \mathcal{Y}\|^{-1} (\mathcal{\boldsymbol{I}} - \mathbf{n}\mathbf{n}^T) \frac{D(\nabla \mathcal{Y})}{Dt}
\label{eq:trans_normals}
\end{align}
%
The equations \eqref{eq:trans_magn_grad_c}, \eqref{eq:trans_thickness} and  \eqref{eq:trans_normals}
depend on the transport equation of the scalar gradient, $\nabla \mathcal{Y}$, which is obtained
from \eqref{eq:trans_scalar} as 
%
\begin{equation}
\frac{D(\nabla \mathcal{Y})}{Dt} = \nabla \left( \frac{D\mathcal{Y}}{Dt}\right) + 
\|\nabla \mathcal{Y}\| \nabla \mathbf{u} \cdot\mathbf{n}
\label{eq:trans_scalar_grad_form0}
\end{equation}
%
This equation is often rewritten, by splitting the velocity gradient tensor into tensors of strain-rate, 
$\boldsymbol{\mathcal{S}}  = \frac{1}{2}\left(\nabla \mathbf{u} + \nabla \mathbf{u}^{T}\right)$, and rotation-rate, $\boldsymbol{\Omega}  = \frac{1}{2}\left(\nabla \mathbf{u} - \nabla \mathbf{u}^{T}\right)$, according to the Hermitian/skew-Hermitian Cauchy-Stokes decomposition as
%
\begin{dmath}
\frac{D(\nabla \mathcal{Y})}{Dt} = \nabla \left( \frac{D\mathcal{Y}}{Dt}\right) +
                   \|\nabla \mathcal{Y}\| \left(
                        \boldsymbol{\mathcal{S}} \cdot \mathbf{n} + 
                        \boldsymbol{\Omega} \cdot \mathbf{n}
                        \right) \\
       = \nabla \left( \frac{D\mathcal{Y}}{Dt}\right) +
        \|\nabla \mathcal{Y}\| \left(
        \boldsymbol{\mathcal{S}} \cdot \mathbf{n}
        +\frac{1}{2} \mathbf{n} \times \boldsymbol{\omega}
        \right)
\label{eq:cauchy_stokes}
\end{dmath}
%
where $\boldsymbol{\omega} = \nabla \times \mathbf{u}$ denotes the vorticity vector.
%
This form was widely used in prior studies, see for instance \cite{hamlington2011interactions,wacks2016flow,zhao2018dynamics}, to scrutinize the interactions of the small-scale turbulence, characterized by $\boldsymbol{\mathcal{S}}$ and $\boldsymbol{\omega}$, and the flame front.
%
Indeed, as highlighted by the equations above, the terms perpendicular (parallel) to the flame
normals, are expected to directly influence the dynamics of the normals (local width).
%
Consequently, both the strain-rate and the vorticity terms in the equation \eqref{eq:cauchy_stokes}
play a role in the dynamics of the flame wrinkling and orientation, whereas only the strain-rate
term affects $\delta_t$.
%
In light of the above, the turbulence-flame interactions are reflected by the alignments of flame 
normals with the normalized vorticity vector $\hat{\boldsymbol{\omega}} = \boldsymbol{\omega}/\|\boldsymbol{\omega}\|$ and with the principal directions $\mathbf{e}_i$ of the strain-rate tensor $\boldsymbol{\mathcal{S}}$.
%
The alignments are given with the absolute value of the cosine of the angle between two vectors,
such as the value 1 indicates a perfect alignment and 0 refer to a perfect misalignment 
(perpendicularity).
%
Given the symmetry of the strain-rate tensor, the eigenvalues $\mathbf{e}_i$, correspond to the
three real eigenvalues, $\lambda_i$, of $\boldsymbol{\mathcal{S}}$ ordered as $
\lambda_1>\lambda_2>\lambda_3$, where $\mathbf{e}_1$, $\mathbf{e}_2$ and $\mathbf{e}_3$ can be 
referred to as the extensive, intermediate and compressive directions.
%
One of the key findings of this family of analysis, conducted either on the passive and reactive 
scalars with passive chemical reactions or flame characterized with a high Karlovitz number,  
is preferential alignment of the scalar gradient with the most compressive direction 
$\mathbf{e}_3$ \cite{ashurst1987alignment,rutland1993direct,kim2007scalar,chakraborty2007influence,hamlington2011interactions,hartung2008effect,swaminathan2006interaction}.
%
However, this behaviour is altered in weak to moderate turbulence (low Karlovitz) where the 
flame-induced thermal expansion affects the turbulence-scalar interaction and the normals are
aligned with the most extensive direction $\mathbf{e}_1$ \cite{chakraborty2007influence,
cifuentes2014local,chakraborty2007comparison}.\\


Obviously, as demonstrated above, the classical decomposition of the velocity tensor has led
to important conclusions in terms of turbulence-flame interactions.
%
However, this decomposition obscures the understanding of some of the processes taking place in the
interaction of the flame with the turbulent flow field.
%
Indeed, as highlighted by a certain number of recent studies \cite{kolavr2007vortex,gao2019rortex,nagata2020triple}, the Hermitian/skew-Hermitian decomposition is constrained by its inability
to segregate pure local normal-straining and rigid-body-rotation effects from non--local dynamics
induced by shearing, as the latter is an intrinsic part of both strain-rate and vorticity tenors.
%
More recently, by performing a complex Schur decomposition of $\nabla \mathbf{u}$,
\citet{keylock2018schur}, dissociated the shear contribution from the other dynamics of the 
velocity gradient.
%
In this framework, the decomposition introduced by \citet{keylock2018schur}, isolates the local
dynamics, which are associated to the eigenvalues, from the non-local effects induced by the vortical
structures following
%
\begin{equation}
\boldsymbol{\mathcal{P}}^* \nabla \mathbf{u} \boldsymbol{\mathcal{P}} =  \mathcal{T} 
            = \boldsymbol{\mathcal{D}_{\lambda}}  +  \boldsymbol{\mathcal{N}}
\label{eq:complex_schur}
\end{equation} 
%
where $(.)^*$ is the Hermitian operator, $\boldsymbol{\mathcal{P}} \in \mathbb{C}^{(3\times3)}$ is a unitary
matrix, \ie $\boldsymbol{\mathcal{P}\mathcal{P}}^*=\mathbf{I}$ and $\boldsymbol{\mathcal{T}} \in \mathbb{C}^{(3\times3)}$ is an upper-triangular matrix.
%
The matrix $\boldsymbol{\mathcal{D}_{\lambda}}$ is the diagonal matrix composed of $\nabla \mathbf{u}$ eigenvalues and $\boldsymbol{\mathcal{N}}$ is a strictly upper-triangular matrix such that
%
\begin{equation}
\boldsymbol{\mathcal{D}_{\lambda}} =   
\begin{pmatrix}
\lambda_1 & 0 & 0\\ 
0         & \lambda_2    & 0\\
0         & 0            & \lambda_3
\end{pmatrix}
%
~~~~~~,~~~~~
\boldsymbol{\mathcal{N}} = \begin{pmatrix}
0 & \gamma_{12} & \gamma_{13}\\ 
0         & 0    & \gamma_{23}\\
0         & 0            & 0
\end{pmatrix}
\end{equation}
where $\lambda_i$ refer to the eigenvalues of the velocity gradient tensor ordered in the 
increasing order of their norms.


The approach based on the equation \eqref{eq:complex_schur} presents the advantage of projecting the
non-normal (non-local) effects into an upper-triangular matrix that can be treated in an analogous
manner to the analysis based on principal eigenvectors, which eases the interpretation of the
respective effects of normal and non--normal contributions.
%
as a matter of fact, the concept embodied by the equation \eqref{eq:complex_schur} has proven to be 
an effective tool to provide concrete explanations concerning the alignments properties, such
as the one between the vorticity vector and the intermediate strain eigenvector and demonstrates
the key role played by non-normality \cite{keylock2018schur}.


The present work aims at reinterpreting some of the classical results obtained in 
the context of turbulence-flame interaction, in the light of concept presented above, and seeking
a deeper insight and further clarity into the phenomena taking place in a such interaction.
%
In view of this, the transport equation of the scalar gradient \eqref{eq:cauchy_stokes}
is re-expressed as:
%
\begin{equation}
\frac{D(\nabla \mathcal{Y})}{Dt} = \nabla \left( \frac{D\mathcal{Y}}{Dt}\right) +
        \|\nabla \mathcal{Y}\| \left(
        \boldsymbol{\mathcal{A_L}} \cdot \mathbf{n}
        +\boldsymbol{\mathcal{A_N}} \cdot \mathbf{n}
        \right)
\label{eq:trans_grad_schur}
\end{equation}
%
where $\boldsymbol{\mathcal{A_L}} \equiv \boldsymbol{\mathcal{P}} \boldsymbol{\mathcal{D}_{\lambda}}
\boldsymbol{\mathcal{P}}^*$ and $\boldsymbol{\mathcal{A_N}} \equiv \boldsymbol{\mathcal{P}} \boldsymbol{\mathcal{N}} \boldsymbol{\mathcal{P}}^*$  denote respectively the local (normal) and non-local 
(non-normal) components of the velocity gradient.
