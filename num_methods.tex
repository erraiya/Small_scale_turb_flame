\section{Governing equations and numerical methods}

The numerical simulations in this study are solved within the framework of the low Mach number (LMN) formulation of the fully compressible reacting flow Navier--Stokes equations. 
%
The consideration of this approximation make it possible to decouple the flow thermo-chemical state
from the momentum evolution \cite{paolucci1982filtering}. 
%
Indeed, in this regime, the evolution of the advection, diffusion and reaction processes takes
place on a spatially homogeneous background pressure manifold.
%
Accordingly, the pressure field is decomposed into a sum of (i) a spatially homogeneous part $p^{(0)}(t)$, which is used to define the thermodynamic state of the fluid mixture, and (ii) a hydrodynamic part 
$\pi(\mathbf{x}, t)$ which satisfies $\pi/p^{0} \sim \mathcal{O}(\mathcal{M}^2)$, where $\mathcal{M}$
denotes the Mach number.
%
The hydrodynamic pressure $\pi$ acts as fluid motion-consistent Lagrange multiplier, whereas the
 $p^0$ is constant in the frame of the fluid.
%
Within this framework, the fluid is treated as a perfect gases mixture of $N_{sp}$ species
and the conservation of momentum as well as the dynamics of species mass fractions 
and mixture enthalpy are given by the following transport equations

\begin{align}
\pd{\rho \mathbf{u}}{t} + \nabla \cdot (\rho \mathbf{u} \mathbf{u} + \boldsymbol{\tau})
&= -\nabla \pi  + \rho \boldsymbol{\Gamma},
\label{eq:lmn_mom}\\
\pd{\rho  Y_k}{t} +  \nabla \cdot  (\rho  Y_k \mathbf{u} + \boldsymbol{\mathcal{F}}_k)
&=  \dot{\omega}_k ,
\label{eq:lmn_spec}\\
\pd{\rho h}{t} + \nabla \cdot  (\rho h \mathbf{u} + \boldsymbol{\mathcal{Q}}) 
&= 0
\label{eq:lmn_enth}
\end{align}
%
where $\rho$ is the density, $\mathbf{u}$ the fluid velocity, $Y_k$, $\boldsymbol{\mathcal{F}}_k$,
 $\dot{\omega}_k$ denote respectively the mass fraction, the diffusive flux and the net reaction 
 rate of the $k$\th,  $h$ is the mass-weighted enthalpy of the mixture, 
 \ie $h=Y_k h_k$, with $h_k$ the enthalpy of the species $k$, $\boldsymbol{\mathcal{Q}}$ designates 
 the heat flux and $\boldsymbol{\Gamma}$ is an external forcing term.

The system of equations \eqref{eq:lmn_mom}-\eqref{eq:lmn_enth} is supplemented by the perfect 
gas equation of state (EOS) $p^{(0)}=\rho \mathcal{R} T / W$, where the mixture molecular weight
 $W$ is given by $W= \sum_{k=1}^{\mathcal{N}_{\mathrm{sp}}} W_k/Y_{k}$, with $W_{k}$ being 
the molecular weight of the species $k$ and $\mathcal{R}$ the the universal gas constant.


The fluid is supposed Newtonian and the bulk viscosity effects are ignored, thus the stress tensor
$\boldsymbol{\tau}$ is expressed as

\begin{equation}
\mathbf{\tau}=\mu\left(\nabla\mathbf{u}+ \nabla\mathbf{u}^T\right)- \frac{2}{3}
\mu\left(\nabla\cdot\mathbf{u}\right)\mathbf{I},
\end{equation}
%
where $\mathbf{I}$ is the identity tensor and $\mu$ is the mixture dynamic viscosity.
%
The heat flux $\boldsymbol{\mathcal{Q}}$ takes the form 
%
\begin{equation}
    \boldsymbol{\mathcal{Q}} =  \sum_{k=1}^{\mathcal{N}_{\mathrm{sp}}} h_k \boldsymbol{\mathcal{F}}_{k}  - \lambda \nabla T
\end{equation}
%
where $\lambda$ is the mixture thermal conductivity.
%
A mixture-averaged model ignoring Soret, Dufour and barodiffusion effects is retained 
and the diffusive flux for the $k$\th species reads
\begin{equation}
    \boldsymbol{\mathcal{F}}_k = -\rho \mathcal{D}_k Y_k \nabla X_k
\end{equation}
%
with $X_k$ is the k\th $\mathcal{D}_k$ is the species mixture-averaged diffusion coefficient.

The diffusion coefficient $\mathcal{D}_k$ is obtained using a mixing rule combining binary diffusion coefficients of all mixture species \cite{hirschfelder1954molecular}.
%
A similar approach is adopted for the evaluation of the mixture thermal conductivity, 
$\lambda(T, Y_k)$ and viscosity $\nu(T, Y_k)$ which is based upon the mixing formulations
of \cite{mathur1967thermal} and \cite{wilke1950viscosity} respectively.


The LMN set of transport equations is solved on a Cartesian grid with constant grid spacing, using the finite volume, low-Mach, adaptive mesh \texttt{PeleLM} solver developed at Lawrence Berkeley National Laboratory \cite{almgren1998conservative,day2000numerical,nonaka2012deferred,pazner2016high,nonaka2018conservative} and built on top of the block-structured AMR \texttt{AMReX} 
\cite{zhang2019amrex} .
%
Over the past two decades, the open-source solver, \texttt{PeleLM}, have been repeatedly validated on numerous inert and reactive configurations.
%
For instance, \texttt{PeleLM} was used in a similar context to the current DNS database, to characterize low-Mach, high-Karlovitz turbulent flames, see \citet{aspden2015turbulence,aspden2016three,aspden2017turbulence,aspden2018towards}.

The numerical procedure implemented in \texttt{PeleLM} are presented in much
greater detail in \cite{nonaka2018conservative}.
%
However, in the following, we restrict ourselves to a global overview of the solver.
% 

The overall numerical method is second-order accurate in time and space and rests on a 
a coupling of the transport and chemistry with a density-weighted approximate projection 
method through a multi-implicit spectral deferred correction strategy \cite{pember1998adaptive,almgren1998conservative, day2000numerical,dutt2000spectral,nonaka2012deferred,pazner2016high, nonaka2018conservative} where each process (advection, diffusion and reaction) is advanced in time
thanks to a lagged approximation of the others.
%
The latter is introduced as an auxiliary time-dependent forcing term.
%
The advantage presented by this procedure consists in enforcing the constrained evolution of the
velocity field while maintaining mass and enthalpy conservation and satisfying the EOS.
%
This constraint,obtained by re-expressing the continuity equation using the EOS, is imposed in
 the form of the flow velocity divergence such that
%
\begin{dmath}
\nabla \cdot \boldsymbol{u} = \frac{1}{T}\frac{DT}{Dt}+ W \sum_k \frac{1}{W_k} \frac{DY_k}{Dt}\\
                            = (\rho c_p T)^{-1}\left[ \nabla \cdot (\lambda \nabla T)
                                                         + \sum_k 
                                                           \left(
                                                                 \rho \mathcal{D}_k
                                                                 \nabla h_k \cdot  \nabla Y_k 
                                                            \right) 
                                                   \right] \\
                           ~~~~~+ \rho^{-1}~~~~ \left[ 
                                    \sum_k \frac{W}{W_k} \nabla \cdot (\rho \mathcal{D}_k \nabla Y_k) +
                                    \sum_k \left(\frac{W}{W_k} - \frac{h_k}{c_p T}\right) \dot{\omega}_k
                               \right]
\label{eq:div_constraint}
\end{dmath}
%
where the operator $D\cdot/Dt$ is the material derivative, $T$ refer to mixture temperature and $c_p=\partial h/\partial T$ is the specific heat of the mixture at constant pressure.


The advective fluxes are treated using a time explicit, second order, unsplit Godunov scheme with
upwinding to compute an advection velocity field $u^{adv,*}$, which is unconstrained according
to ~\eqref{eq:div_constraint}.
%
A density-weighted approximate projection applied to $u^{adv,*}$, decomposes the unconstrained advection
velocity into the hydrodynamic pressure and a velocity field that satisfies the constraint 
~\eqref{eq:div_constraint}.
%
The latter is retained as the solution velocity field.


A semi-implicit discretization with a generalized Crank-Nicolson scheme is used to advance the
diffusion, whereas the backward difference method is considered to advance the chemistry.
%
The integration time step is determined on the basis of the Courant–Friedrichs–Levy (CFL)
considerations.
%
The thermodynamic coefficients as well as the species reaction rates in the same fashion as the
\texttt{CHEMKIN} \cite{kee1990chemkin} specification, whereas \texttt{EGLIB}\cite{ern2004eglib} is
used to evaluate the mixture-averaged transport properties.
