%%%%%%%%%%%%%%%%%%%%%%%%%%%%%%%%%%%%%%%%%%%%%%%%%%%%%%%%%%%%%%%%%%%%%%%%%%%%%
%
%                 Packages and Commands use for the document
%
%%%%%%%%%%%%%%%%%%%%%%%%%%%%%%%%%%%%%%%%%%%%%%%%%%%%%%%%%%%%%%%%%%%%%%%%%%%%%
% utilisez la commande "%%%%%%%%%%%%%%%%%%%%%%%%%%%%%%%%%%%%%%%%%%%%%%%%%%%%%%%%%%%%%%%%%%%%%%%%%%%%%
%
%                 Packages and Commands use for the document
%
%%%%%%%%%%%%%%%%%%%%%%%%%%%%%%%%%%%%%%%%%%%%%%%%%%%%%%%%%%%%%%%%%%%%%%%%%%%%%
% utilisez la commande "%%%%%%%%%%%%%%%%%%%%%%%%%%%%%%%%%%%%%%%%%%%%%%%%%%%%%%%%%%%%%%%%%%%%%%%%%%%%%
%
%                 Packages and Commands use for the document
%
%%%%%%%%%%%%%%%%%%%%%%%%%%%%%%%%%%%%%%%%%%%%%%%%%%%%%%%%%%%%%%%%%%%%%%%%%%%%%
% utilisez la commande "%%%%%%%%%%%%%%%%%%%%%%%%%%%%%%%%%%%%%%%%%%%%%%%%%%%%%%%%%%%%%%%%%%%%%%%%%%%%%
%
%                 Packages and Commands use for the document
%
%%%%%%%%%%%%%%%%%%%%%%%%%%%%%%%%%%%%%%%%%%%%%%%%%%%%%%%%%%%%%%%%%%%%%%%%%%%%%
% utilisez la commande "\input{config.tex}" dans le preambule du document
% str="chaine de caractere a chercher" ; grep -rni --include=\*.{tex,bib} "$str" .



% Main layouts
% %%%%%%%%%%%%%%%%%%%%%%%%%%%%%%%%%%%%%%%%%%%%%%%%%%%%%%%%%%%%%%%%%%%%%%%%%%%%%
% \usepackage[utf8]{inputenc}
% \usepackage[T1]{fontenc}       % fontes TeX pour la césure des mots et accents PDF
% \usepackage{xspace} 
% \usepackage{textcomp} % symboles divers (autres que mathématiques, ex: degree C) (OBSOLETE 2017 -> ça dépend quelle font est utilisée)
\pdfmapfile{=md-chr7v.map}
\usepackage[super]{nth}
\newcommand{\up}[1]{\textsuperscript{#1}}
\newcommand{\second}{2\up{nd}\xspace}
\newcommand{\seconde}{2\up{nde}\xspace}


% For the PDF document
%%%%%%%%%%%%%%%%%%%%%%%%%%%%%%%%%%%%%%%%%%%%%%%%%%%%%%%%%%%%%%%%%%%%%%%%%%%%%
\usepackage{hyperref}
%\usepackage%[backref=page]
%   {hyperref}
    \hypersetup{
        backref=true,       % permet d'ajouter des liens dans...
        pagebackref=true,   % ...les bibliographies
        hyperindex=true,    % ajoute des liens dans les index.
        colorlinks=true,    % colorise les liens
        breaklinks=true,    % permet le retour a la ligne dans les liens trop longs
        urlcolor= blue,     % couleur des hyperliens
        linkcolor= blue,    % couleur des liens internes
        citecolor=red,      % couleur des citations
    }


% % Comment big section
% %%%%%%%%%%%%%%%%%%%%%%%%%%%%%%%%%%%%%%%%%%%%%%%%%%%%%%%%%%%%%%%%%%%%%%%%%%%%%
% \usepackage{verbatim}
% \usepackage{setspace}


% % Mathematics
% %%%%%%%%%%%%%%%%%%%%%%%%%%%%%%%%%%%%%%%%%%%%%%%%%%%%%%%%%%%%%%%%%%%%%%%%%%%%%
% % \usepackage{amsmath,amssymb,amsfonts,amsbsy} % amsthm
% % \usepackage[version=4]{mhchem}
% % \usepackage{systeme,mathtools}
% \usepackage{amsfonts} % amsthm
\usepackage{xspace}
% \usepackage{amsmath} % amsthm
% % \usepackage[T1]{fontenc}
% \usepackage{lmodern}
% \usepackage{mathtools}% http://ctan.org/pkg/mathtools


\usepackage{graphicx}
\usepackage{amsmath}
\usepackage[version=4]{mhchem}
\usepackage{siunitx}
\usepackage{longtable,tabularx}
% \setlength\LTleft{0pt} 
\usepackage{amsfonts}

% % % Units
% %%%%%%%%%%%%%%%%%%%%%%%%%%%%%%%%%%%%%%%%%%%%%%%%%%%%%%%%%%%%%%%%%%%%%%%%%%%%%
% \usepackage{siunitx}             % unités physiques
% %\usepackage[autolanguage,np]{numprint}
% % séparateur milliers ; symbole des décimales '.' (UK) ou ',' (FR)
% % exemple : \np[N/mm^2]{-123456.1234e3}


% % Tables
% %%%%%%%%%%%%%%%%%%%%%%%%%%%%%%%%%%%%%%%%%%%%%%%%%%%%%%%%%%%%%%%%%%%%%%%%%%%%%
% \usepackage{tabularx}
% \usepackage{booktabs}  
% % \toprule, \midrule, \bottomrule, \addlinespace[<largeur>], ...
% % do not used anymore \hline or |
% % can be used for equation in tables !!!
% \renewcommand{\toprule}{\hline\hline\addlinespace[2pt]}
% \renewcommand{\bottomrule}{\addlinespace[2pt]\hline\hline}
% \usepackage{multirow}  % fusionner lignes d'un tableau
% \usepackage{longtable} % table sur plusieurs pages
% %\usepackage{extra_packages/tabu} % replace tabularx

% \usepackage{threeparttable} % footnote table
% \renewcommand{\tnote}[1]{\textsuperscript{\textcolor{blue}{(\TPTtagStyle{#1})}}}



% % Figures
% %%%%%%%%%%%%%%%%%%%%%%%%%%%%%%%%%%%%%%%%%%%%%%%%%%%%%%%%%%%%%%%%%%%%%%%%%%%%%
% \usepackage{epstopdf}
% \usepackage{rotating} % sideways
% \usepackage{graphicx}
% \usepackage{graphbox} % ex­ten­sion of graph­icx (vertical positionning)
% \usepackage{grffile}             % add file names contain several dots, e.g. "file.name.simulation.pdf"
% \usepackage[percent]{overpic}    % superposer du texte sur image

% %\usepackage{printlen} % print length in human
% %   \uselengthunit{cm}

% \newlength{\mywidth}
% \newlength{\myheight}
% \newlength{\mywidthgraph}
% \newlength{\myheightgraph}
% \newlength{\myheightgraphb}
% \newlength{\mywidthsubfig}

% % === Subfigure ===
% %\usepackage{subfigmat}  % last release 1999 (OBSOLETE...)
% %\usepackage{subfigure}  % last release 2002 (OBSOLETE...)
% %\usepackage{subfig}     % last release 2005 (can be used...)
% \usepackage[
% hypcap=true, 
% %   subrefformat=parens, 
% labelfont=normalfont, % up/it/sl/sc/md/bf(default)/rm/sf/tt/...
% labelformat=simple, 
% ]{subcaption} % last release 2013 INCOMPATIBLE with subfigure or subfig
% \renewcommand\thesubfigure{\bf\alph{subfigure})} % format in caption
% \newcommand{\sublabelsty}[1]{\bf\small{#1)}}


% % Listes, items, descriptions
% %%%%%%%%%%%%%%%%%%%%%%%%%%%%%%%%%%%%%%%%%%%%%%%%%%%%%%%%%%%%%%%%%%%%%%%%%%%%%

% %% IF you use \usepackage[french/frenchb/francais]{babel}, include :
% %% http://daniel.flipo.free.fr/frenchb/frenchb-doc.pdf
% %\frenchbsetup{
% %%   StandardLists=true, % 
% %   StandardItemLabels=true,
% %%   ReduceListSpacing=true,
% %%   CompactItemize=true,
% %   }

% % === PERSONALISER DES LISTES ===
% %\usepackage[olditem,oldenum,defblank,neveradjust]{paralist} % last release 2017
% %\setlength{\plitemsep}{0.5em}
% % http://tug.ctan.org/macros/latex/contrib/paralist/paralist.pdf
% %             ENUMERATE     ITEMIZE       DESCRIPTION
% %   AS PARA   asparaenum    asparaitem    asparadesc
% %   IN PARA   inparaenum    inparaitem    inparadesc
% %   COMPACT   compactenum   compactitem   compactdesc
% % AVOID USING itemize and enumerate environment...
% %
% % === PUIS === (sinon conflit)
% %\usepackage[shortlabels]{enumitem} % last release 2011 (PUT IT AFTER paralist)
% % https://tex.stackexchange.com/questions/176122/problem-with-indent-in-enumeration-of-paralist
% % http://www.edu.upmc.fr/c2i/ressources/latex/aide-memoire.pdf




% %%%%%%%%%%%%%%%%%%%%%%%%%%%%%%%%%%%%%%%%%%%%%%%%%%%%%%%%%%%%%%%%%%%%%%%%%%%%%
% % RACCOURCIS MATHEMATIQUES
% %%%%%%%%%%%%%%%%%%%%%%%%%%%%%%%%%%%%%%%%%%%%%%%%%%%%%%%%%%%%%%%%%%%%%%%%%%%%%

% % === Operateurs AVEC arguments ===
% \newcommand{\dd}{\mathop{}\mathopen{}\mathrm{d}}                  % differentielle d
% \newcommand{\td}[3][]{\frac{\dd^{#1}#2}{\dd{#3}^{#1}}}            % total derivative

% \newcommand{\pd}[3][]{\frac{\partial^{#1}#2}{\partial{#3}^{#1}}}  % partial derivative                            % << NOTE >> : 
% % devrait inverser l'argument #2 et #3 pour pouvoir si besoin juste faire \partial_j{f} = \pd{x_j}{f}
% % désignerait alors juste l'opérateur de dérivée partielle. Idem pour tous les autres...

% \newcommand{\pDelta}[3][]{\dfrac{\delta^{#1}#2}{\delta{#3}^{#1}}} % partial derivative (discrete form)

% \newcommand{\pD}[3][]{\dfrac{D^{#1}#2}{D{#3}^{#1}}}               % particular derivative
% \newcommand{\abs}[1]{\left\lvert #1 \right\rvert}                 % |.| absolute value
% \newcommand{\norme}[1]{\| #1 \|}                                  % ||.|| small norme
% \newcommand{\Norme}[1]{\left\| #1 \right\|}                       % ||.|| large norme
% \newcommand{\lremp}[1]{\left. #1 \right.}                         %  . (empty)
% \newcommand{\lrbkt}[1]{\left( #1 \right)}                         % (.)
% \newcommand{\lrsbkt}[1]{\left[ #1 \right]}                        % [.]
% \newcommand{\lrangle}[1]{\left\langle #1 \right\rangle}           % <.>
% %\newcommand{\blrangle}[2]{#1\langle #2 #1\rangle}                 % <.> big,Big,...
% \newcommand{\lrbrace}[1]{\left\lbrace #1 \right\rbrace}           % {.}
% \newcommand{\blrbrace}[1]{\big\lbrace #1 \big\rbrace}             % {.} big

% %\newcommand{\bs}[1]{\pmb{#1}}                                     % bold symbol
% \newcommand{\bs}[1]{\boldsymbol{#1}}                              % bold symbol (doesn't work with FOURIER font and \nabla)
% %\renewcommand{\vec}{\bs}

% \renewcommand{\tilde}{\widetilde}                                 % ~
% \renewcommand{\hat}{\widehat}                                     % ^
% \renewcommand{\bar}{\overline} 
\newcommand{\pars}[1]{\left(\,{#1}\,\right)}
%                              % -
\usepackage[super]{nth}

% %\newcommand{\range}[4]{\mathopen{}\left#1 #2 \mathpunct{} , #3 \mathclose{}\right#4}
% \newcommand{\range}[4]{\mathopen{}\left#1 #2 ,\: #3 \mathclose{}\right#4}
% \newcommand{\rangeCC}[2]{\range{[}{#1}{#2}{]}}                    % range Close-Close
% \newcommand{\rangeOC}[2]{\range{]}{#1}{#2}{]}}                    % range Open -Close
% \newcommand{\rangeCO}[2]{\range{[}{#1}{#2}{[}}                    % range Close-Open
% \newcommand{\rangeOO}[2]{\range{]}{#1}{#2}{[}}                    % range Open -Open
% \newcommand{\rangeI} [2]{\range{\llbracket}{#1}{#2}{\rrbracket}}  % range Integer

% \newcommand{\colprod}[1]{\textcolor{red}{#1}}                     % COLOR production
% \newcommand{\coldissip}[1]{\textcolor{DeepSkyBlue3}{#1}}          % COLOR dissipation


% % === Operateurs SANS arguments ===
% \DeclareMathOperator\erf{\mathrm{erf}}          % error function
% \DeclareMathOperator\Real{\mathrm{Re}}          % real part
% \DeclareMathOperator\Imag{\mathrm{Im}}          % imaginary part

% \DeclareMathOperator\diverg{\mathrm{div}}       % divergence
% \DeclareMathOperator\rot{\vec{\mathrm{rot}}}    % rotationel
% \DeclareMathOperator\grad{\vec{\mathrm{grad}}}  % gradient

% \DeclareMathOperator{\Lsource}{\mathcal{L}}     % source/sink filtered op
% \DeclareMathOperator{\Lsourced}{\mathit{L}}     % idem for discrete case

% \newcommand\eqdef{\overset{\mathrm{def}}{=}}    % = def
% \newcommand\gdo{\mathcal{O}}                    % grand O
% %\newcommand{\rar}{\rightarrow}                  % ->
% %\newcommand{\rlha}{\rightleftharpoons}          % <=>

% % === LISTE DES VARIABLES, INDICES, EXPOSANTS ===
% % Regle d'uniformisation de syntaxe : 
% %   Nom de la variable = nom en anglais abrégé
% % http://pleasemakeanote.blogspot.fr/2010/07/italics-in-math-equations.html
% % http://folk.uio.no/jornb/howto/latex/jfm2enot.pdf
% \newcommand{\inmath}[1]{\ensuremath{#1}\xspace}
% \newcommand{\inmathrm}[1]{\ensuremath{\mathrm{#1}}\xspace}
% \newcommand{\inmathit}[1]{\ensuremath{\mathit{#1}}\xspace}


% \newcommand\domain{\mathcal{D}}              % Domain
% \newcommand\cst{\inmathrm{cst}}              % Constant
% \newcommand\fuel{\inmathrm{fuel}}            % Fuel stream
% \newcommand\ox{\inmathrm{ox}}                % Oxydizer stream
% \newcommand\fsp{\inmath{F}}                  % Fuel species
% \newcommand\osp{\inmath{O}}                  % Oxydizer species
% \newcommand\psp{\inmath{P}}                  % Product species
% \newcommand\strainrate{\inmath{a}}           % Strain rate
% \newcommand\SDR{\inmath{\chi_\xi}}           % Scalar Dissipation Rate (SDR)
% \newcommand\sgsSDR{\inmath{\chi_\xi^\sgs}}   % SGS Scalar Dissipation Rate
% \newcommand\stc{\mathrm{st}}                 % Stoechiometry
% \newcommand\stoecr{\inmath{r_\stc}}          % Stoechiometric ratio
% \newcommand\mostr{\mathrm{mr}}               % Most Reactive
% \newcommand\res{\mathrm{res}}                % Resolved
% \newcommand\sgs{\mathrm{sgs}}                % Sub-Grid Scale
% \newcommand\rms{\mathrm{rms}}                % Root-mean-square
% \newcommand\PSR{\inmathrm{PSR}}              % Perfectly Stirred Reactor
% \newcommand\MIL{\inmathrm{MIL}}              % Model Intermittent Lagrangian
% \newcommand\mix{\mathrm{mix}}                % Pure Mixing
% \newcommand\NPR{\inmathit{NPR}}              % Nozzle Pressure Ratio
% \newcommand\IEM{\inmathrm{IEM}}              % Interaction by Exchange with Mean
% \newcommand\traceless{\inmath{d}}            % Deviatrice / Traceless
% \newcommand\seuil{\inmathrm{seuil}}          % Seuil
% \newcommand\eq{\mathrm{eq}}                  % Equilibrium
% \newcommand\diff{\mathrm{diff}}              % Diffusion
% \newcommand\conv{\mathrm{conv}}              % Convection
% \newcommand\igni{\mathrm{igni}}              % Ignition
% \newcommand\propa{\mathrm{propa}}            % Propagation
% \newcommand\unburnt{\inmathrm{u}}            % Unburnt
% \newcommand\burnt{\inmathrm{b}}              % Burnt
% \newcommand\premixed{\mathit{P}}             % Premixed
% \newcommand\npremixed{\mathit{NP}}           % Non-Premixed
% \newcommand\rf{\mathrm{ref}}                 % Reference
% \newcommand\wall{w}                          % Wall
% \newcommand\jet{\mathrm{jet}}                % Jet injection
% \newcommand\SSp{\mathit{SS}}                 % Steady state parameter

% \newcommand\sgsVar{\inmath{\tilde V_\xi}}    % SGS variance
% \newcommand\sqgradzm{\inmath{\vec\nabla\tilde\xi\cdot\vec\nabla\tilde\xi}} % Squared gradient scalar

% % === Amelioration ecriture \tilde{\xi}^2
% %\newcommand\ZtildeS{\inmath{\left.\tilde{\xi}\right.^{2}}} 
% \newcommand\ZtildeS{\inmath{\tilde{\xi\mkern 0mu}^{2}}}
% %\newcommand\ZtildeS{\inmath{\tilde{\xi}{\mathstrut}^{\,2}}}




% % Adimensional numbers
% %%%%%%%%%%%%%%%%%%%%%%%%%%%%%%%%%%%%%%%%%%%%%%%%%%%%%%%%%%%%%%%%%%%%%%%%%%%%%
% \newcommand\adimn[1]{\inmathrm{#1}}   % Font for adimensionalize number

% \newcommand\Rey{\adimn{Re}}  % Reynolds number, cf TeX's \Re real part
% \newcommand\Pran{\adimn{Pr}} % Prandtl number, cf TeX's \Pr product
% \newcommand\Pen{\adimn{Pe}}  % Peclet number
% \newcommand\Sc{\adimn{Sc}}   % Schmidt number
% \newcommand\Str{\adimn{S}}   % Strouhal number
% \newcommand\Le{\adimn{Le}}   % Lewis number
% \newcommand\Ma{\adimn{Ma}}   % Mach number
% \newcommand\Kn{\adimn{Kn}}   % Knudsen number
% \newcommand\Ec{\adimn{Ec}}   % Eckert number
% \newcommand\Fr{\adimn{Fr}}   % Froude number
% \newcommand\Fo{\adimn{Fo}}   % Fourier number
% \newcommand\Ri{\adimn{Ri}}   % Richardson number
% \newcommand\Gra{\adimn{Gr}}  % Grashof number
% \newcommand\Ray{\adimn{Ra}}  % Rayleigh number
% \newcommand\St{\adimn{St}}   % Stanton number
% \newcommand\Nu{\adimn{Nu}}   % Nusselt number
% \newcommand\Da{\adimn{Da}}   % Damköhler number
% \newcommand\Ka{\adimn{Ka}}   % Karlovitz number
% \newcommand\CFL{\adimn{CFL}} % Courant-Friedrichs-Lewy number



% % Cross references
% %%%%%%%%%%%%%%%%%%%%%%%%%%%%%%%%%%%%%%%%%%%%%%%%%%%%%%%%%%%%%%%%%%%%%%%%%%%%%
\usepackage{cleveref}
% % INCOMPATIBLE with the use character ':' and FRENCH in \label 
% % each of the following has two versions
% % \crefname{environmentname}{singular}{plural}, used at mid-sentence
% % \Crefname{environmentname}{singular}{plural}, used at the beginning of a sentence
% \newcommand{\bcref}[1]{[\cref{#1}]}

% \crefname{section}{Sec.}{Secs.}
% %   \Crefname{section}{Section}{Sections}
% \crefname{chapter}{Sec.}{Secs.}
% \crefname{appendix}{Appendix}{Appendices}

% %   \crefname{equation}{}{}
% \crefname{equation}{Eq.}{Eqs.}
% %   \Crefname{equation}{Equation}{Equations}

% \crefname{table}{Table}{Tables}
% %   \Crefname{table}{Tableau}{Tableaux}

% \crefname{figure}{Fig.}{Figs.}
% \Crefname{figure}{Figure}{Figures}

% \crefname{page}{p.}{p.}


% % Raccourcis pour le corpus
% %%%%%%%%%%%%%%%%%%%%%%%%%%%%%%%%%%%%%%%%%%%%%%%%%%%%%%%%%%%%%%%%%%%%%%%%%%%%%
% \newcommand\etal{\mbox{{et al.}}\xspace}
% \newcommand\apriori{\mbox{{a priori}}\xspace}
% \newcommand\aposteriori{\mbox{{a posteriori}}\xspace}
% \newcommand\etc{etc.\xspace}
% \newcommand\eg{e.g.\xspace}
% \newcommand\ie{i.e.\xspace}
% \newcommand\Xth{\textsuperscript{th}\xspace} % for X variable (\th is already taken)

% \newcommand\code[1]{{\fontfamily{pzc}\selectfont #1}\xspace} % Font for code names


% \newcommand\commAM[1]{{\textcolor{blue}{#1}}}   % commentaires
% \newcommand\commAT[1]{\textit{\textcolor{blue}{(#1)}}}   % commentaires
% \newcommand\commGL[1]{\textit{\textcolor{red}{#1}}}     % commentaires
% \newcommand\corrAT[1]{\textcolor{red}{#1}}              % corrections
% \newcommand\corrlatter[1]{\textcolor{red}{#1}}           % a coriger plus tard...

% \setlength\LTleft{0pt} 




% % Coloring
% \usepackage{xcolor}

% \usepackage{txfontsb}
% \let\myBbbk\Bbbk
% \let\Bbbk\relax
% \usepackage[lite]{mtpro2}" dans le preambule du document
% str="chaine de caractere a chercher" ; grep -rni --include=\*.{tex,bib} "$str" .



% Main layouts
% %%%%%%%%%%%%%%%%%%%%%%%%%%%%%%%%%%%%%%%%%%%%%%%%%%%%%%%%%%%%%%%%%%%%%%%%%%%%%
% \usepackage[utf8]{inputenc}
% \usepackage[T1]{fontenc}       % fontes TeX pour la césure des mots et accents PDF
% \usepackage{xspace} 
% \usepackage{textcomp} % symboles divers (autres que mathématiques, ex: degree C) (OBSOLETE 2017 -> ça dépend quelle font est utilisée)
\pdfmapfile{=md-chr7v.map}
\usepackage[super]{nth}
\newcommand{\up}[1]{\textsuperscript{#1}}
\newcommand{\second}{2\up{nd}\xspace}
\newcommand{\seconde}{2\up{nde}\xspace}


% For the PDF document
%%%%%%%%%%%%%%%%%%%%%%%%%%%%%%%%%%%%%%%%%%%%%%%%%%%%%%%%%%%%%%%%%%%%%%%%%%%%%
\usepackage{hyperref}
%\usepackage%[backref=page]
%   {hyperref}
    \hypersetup{
        backref=true,       % permet d'ajouter des liens dans...
        pagebackref=true,   % ...les bibliographies
        hyperindex=true,    % ajoute des liens dans les index.
        colorlinks=true,    % colorise les liens
        breaklinks=true,    % permet le retour a la ligne dans les liens trop longs
        urlcolor= blue,     % couleur des hyperliens
        linkcolor= blue,    % couleur des liens internes
        citecolor=red,      % couleur des citations
    }


% % Comment big section
% %%%%%%%%%%%%%%%%%%%%%%%%%%%%%%%%%%%%%%%%%%%%%%%%%%%%%%%%%%%%%%%%%%%%%%%%%%%%%
% \usepackage{verbatim}
% \usepackage{setspace}


% % Mathematics
% %%%%%%%%%%%%%%%%%%%%%%%%%%%%%%%%%%%%%%%%%%%%%%%%%%%%%%%%%%%%%%%%%%%%%%%%%%%%%
% % \usepackage{amsmath,amssymb,amsfonts,amsbsy} % amsthm
% % \usepackage[version=4]{mhchem}
% % \usepackage{systeme,mathtools}
% \usepackage{amsfonts} % amsthm
\usepackage{xspace}
% \usepackage{amsmath} % amsthm
% % \usepackage[T1]{fontenc}
% \usepackage{lmodern}
% \usepackage{mathtools}% http://ctan.org/pkg/mathtools


\usepackage{graphicx}
\usepackage{amsmath}
\usepackage[version=4]{mhchem}
\usepackage{siunitx}
\usepackage{longtable,tabularx}
% \setlength\LTleft{0pt} 
\usepackage{amsfonts}

% % % Units
% %%%%%%%%%%%%%%%%%%%%%%%%%%%%%%%%%%%%%%%%%%%%%%%%%%%%%%%%%%%%%%%%%%%%%%%%%%%%%
% \usepackage{siunitx}             % unités physiques
% %\usepackage[autolanguage,np]{numprint}
% % séparateur milliers ; symbole des décimales '.' (UK) ou ',' (FR)
% % exemple : \np[N/mm^2]{-123456.1234e3}


% % Tables
% %%%%%%%%%%%%%%%%%%%%%%%%%%%%%%%%%%%%%%%%%%%%%%%%%%%%%%%%%%%%%%%%%%%%%%%%%%%%%
% \usepackage{tabularx}
% \usepackage{booktabs}  
% % \toprule, \midrule, \bottomrule, \addlinespace[<largeur>], ...
% % do not used anymore \hline or |
% % can be used for equation in tables !!!
% \renewcommand{\toprule}{\hline\hline\addlinespace[2pt]}
% \renewcommand{\bottomrule}{\addlinespace[2pt]\hline\hline}
% \usepackage{multirow}  % fusionner lignes d'un tableau
% \usepackage{longtable} % table sur plusieurs pages
% %\usepackage{extra_packages/tabu} % replace tabularx

% \usepackage{threeparttable} % footnote table
% \renewcommand{\tnote}[1]{\textsuperscript{\textcolor{blue}{(\TPTtagStyle{#1})}}}



% % Figures
% %%%%%%%%%%%%%%%%%%%%%%%%%%%%%%%%%%%%%%%%%%%%%%%%%%%%%%%%%%%%%%%%%%%%%%%%%%%%%
% \usepackage{epstopdf}
% \usepackage{rotating} % sideways
% \usepackage{graphicx}
% \usepackage{graphbox} % ex­ten­sion of graph­icx (vertical positionning)
% \usepackage{grffile}             % add file names contain several dots, e.g. "file.name.simulation.pdf"
% \usepackage[percent]{overpic}    % superposer du texte sur image

% %\usepackage{printlen} % print length in human
% %   \uselengthunit{cm}

% \newlength{\mywidth}
% \newlength{\myheight}
% \newlength{\mywidthgraph}
% \newlength{\myheightgraph}
% \newlength{\myheightgraphb}
% \newlength{\mywidthsubfig}

% % === Subfigure ===
% %\usepackage{subfigmat}  % last release 1999 (OBSOLETE...)
% %\usepackage{subfigure}  % last release 2002 (OBSOLETE...)
% %\usepackage{subfig}     % last release 2005 (can be used...)
% \usepackage[
% hypcap=true, 
% %   subrefformat=parens, 
% labelfont=normalfont, % up/it/sl/sc/md/bf(default)/rm/sf/tt/...
% labelformat=simple, 
% ]{subcaption} % last release 2013 INCOMPATIBLE with subfigure or subfig
% \renewcommand\thesubfigure{\bf\alph{subfigure})} % format in caption
% \newcommand{\sublabelsty}[1]{\bf\small{#1)}}


% % Listes, items, descriptions
% %%%%%%%%%%%%%%%%%%%%%%%%%%%%%%%%%%%%%%%%%%%%%%%%%%%%%%%%%%%%%%%%%%%%%%%%%%%%%

% %% IF you use \usepackage[french/frenchb/francais]{babel}, include :
% %% http://daniel.flipo.free.fr/frenchb/frenchb-doc.pdf
% %\frenchbsetup{
% %%   StandardLists=true, % 
% %   StandardItemLabels=true,
% %%   ReduceListSpacing=true,
% %%   CompactItemize=true,
% %   }

% % === PERSONALISER DES LISTES ===
% %\usepackage[olditem,oldenum,defblank,neveradjust]{paralist} % last release 2017
% %\setlength{\plitemsep}{0.5em}
% % http://tug.ctan.org/macros/latex/contrib/paralist/paralist.pdf
% %             ENUMERATE     ITEMIZE       DESCRIPTION
% %   AS PARA   asparaenum    asparaitem    asparadesc
% %   IN PARA   inparaenum    inparaitem    inparadesc
% %   COMPACT   compactenum   compactitem   compactdesc
% % AVOID USING itemize and enumerate environment...
% %
% % === PUIS === (sinon conflit)
% %\usepackage[shortlabels]{enumitem} % last release 2011 (PUT IT AFTER paralist)
% % https://tex.stackexchange.com/questions/176122/problem-with-indent-in-enumeration-of-paralist
% % http://www.edu.upmc.fr/c2i/ressources/latex/aide-memoire.pdf




% %%%%%%%%%%%%%%%%%%%%%%%%%%%%%%%%%%%%%%%%%%%%%%%%%%%%%%%%%%%%%%%%%%%%%%%%%%%%%
% % RACCOURCIS MATHEMATIQUES
% %%%%%%%%%%%%%%%%%%%%%%%%%%%%%%%%%%%%%%%%%%%%%%%%%%%%%%%%%%%%%%%%%%%%%%%%%%%%%

% % === Operateurs AVEC arguments ===
% \newcommand{\dd}{\mathop{}\mathopen{}\mathrm{d}}                  % differentielle d
% \newcommand{\td}[3][]{\frac{\dd^{#1}#2}{\dd{#3}^{#1}}}            % total derivative

% \newcommand{\pd}[3][]{\frac{\partial^{#1}#2}{\partial{#3}^{#1}}}  % partial derivative                            % << NOTE >> : 
% % devrait inverser l'argument #2 et #3 pour pouvoir si besoin juste faire \partial_j{f} = \pd{x_j}{f}
% % désignerait alors juste l'opérateur de dérivée partielle. Idem pour tous les autres...

% \newcommand{\pDelta}[3][]{\dfrac{\delta^{#1}#2}{\delta{#3}^{#1}}} % partial derivative (discrete form)

% \newcommand{\pD}[3][]{\dfrac{D^{#1}#2}{D{#3}^{#1}}}               % particular derivative
% \newcommand{\abs}[1]{\left\lvert #1 \right\rvert}                 % |.| absolute value
% \newcommand{\norme}[1]{\| #1 \|}                                  % ||.|| small norme
% \newcommand{\Norme}[1]{\left\| #1 \right\|}                       % ||.|| large norme
% \newcommand{\lremp}[1]{\left. #1 \right.}                         %  . (empty)
% \newcommand{\lrbkt}[1]{\left( #1 \right)}                         % (.)
% \newcommand{\lrsbkt}[1]{\left[ #1 \right]}                        % [.]
% \newcommand{\lrangle}[1]{\left\langle #1 \right\rangle}           % <.>
% %\newcommand{\blrangle}[2]{#1\langle #2 #1\rangle}                 % <.> big,Big,...
% \newcommand{\lrbrace}[1]{\left\lbrace #1 \right\rbrace}           % {.}
% \newcommand{\blrbrace}[1]{\big\lbrace #1 \big\rbrace}             % {.} big

% %\newcommand{\bs}[1]{\pmb{#1}}                                     % bold symbol
% \newcommand{\bs}[1]{\boldsymbol{#1}}                              % bold symbol (doesn't work with FOURIER font and \nabla)
% %\renewcommand{\vec}{\bs}

% \renewcommand{\tilde}{\widetilde}                                 % ~
% \renewcommand{\hat}{\widehat}                                     % ^
% \renewcommand{\bar}{\overline} 
\newcommand{\pars}[1]{\left(\,{#1}\,\right)}
%                              % -
\usepackage[super]{nth}

% %\newcommand{\range}[4]{\mathopen{}\left#1 #2 \mathpunct{} , #3 \mathclose{}\right#4}
% \newcommand{\range}[4]{\mathopen{}\left#1 #2 ,\: #3 \mathclose{}\right#4}
% \newcommand{\rangeCC}[2]{\range{[}{#1}{#2}{]}}                    % range Close-Close
% \newcommand{\rangeOC}[2]{\range{]}{#1}{#2}{]}}                    % range Open -Close
% \newcommand{\rangeCO}[2]{\range{[}{#1}{#2}{[}}                    % range Close-Open
% \newcommand{\rangeOO}[2]{\range{]}{#1}{#2}{[}}                    % range Open -Open
% \newcommand{\rangeI} [2]{\range{\llbracket}{#1}{#2}{\rrbracket}}  % range Integer

% \newcommand{\colprod}[1]{\textcolor{red}{#1}}                     % COLOR production
% \newcommand{\coldissip}[1]{\textcolor{DeepSkyBlue3}{#1}}          % COLOR dissipation


% % === Operateurs SANS arguments ===
% \DeclareMathOperator\erf{\mathrm{erf}}          % error function
% \DeclareMathOperator\Real{\mathrm{Re}}          % real part
% \DeclareMathOperator\Imag{\mathrm{Im}}          % imaginary part

% \DeclareMathOperator\diverg{\mathrm{div}}       % divergence
% \DeclareMathOperator\rot{\vec{\mathrm{rot}}}    % rotationel
% \DeclareMathOperator\grad{\vec{\mathrm{grad}}}  % gradient

% \DeclareMathOperator{\Lsource}{\mathcal{L}}     % source/sink filtered op
% \DeclareMathOperator{\Lsourced}{\mathit{L}}     % idem for discrete case

% \newcommand\eqdef{\overset{\mathrm{def}}{=}}    % = def
% \newcommand\gdo{\mathcal{O}}                    % grand O
% %\newcommand{\rar}{\rightarrow}                  % ->
% %\newcommand{\rlha}{\rightleftharpoons}          % <=>

% % === LISTE DES VARIABLES, INDICES, EXPOSANTS ===
% % Regle d'uniformisation de syntaxe : 
% %   Nom de la variable = nom en anglais abrégé
% % http://pleasemakeanote.blogspot.fr/2010/07/italics-in-math-equations.html
% % http://folk.uio.no/jornb/howto/latex/jfm2enot.pdf
% \newcommand{\inmath}[1]{\ensuremath{#1}\xspace}
% \newcommand{\inmathrm}[1]{\ensuremath{\mathrm{#1}}\xspace}
% \newcommand{\inmathit}[1]{\ensuremath{\mathit{#1}}\xspace}


% \newcommand\domain{\mathcal{D}}              % Domain
% \newcommand\cst{\inmathrm{cst}}              % Constant
% \newcommand\fuel{\inmathrm{fuel}}            % Fuel stream
% \newcommand\ox{\inmathrm{ox}}                % Oxydizer stream
% \newcommand\fsp{\inmath{F}}                  % Fuel species
% \newcommand\osp{\inmath{O}}                  % Oxydizer species
% \newcommand\psp{\inmath{P}}                  % Product species
% \newcommand\strainrate{\inmath{a}}           % Strain rate
% \newcommand\SDR{\inmath{\chi_\xi}}           % Scalar Dissipation Rate (SDR)
% \newcommand\sgsSDR{\inmath{\chi_\xi^\sgs}}   % SGS Scalar Dissipation Rate
% \newcommand\stc{\mathrm{st}}                 % Stoechiometry
% \newcommand\stoecr{\inmath{r_\stc}}          % Stoechiometric ratio
% \newcommand\mostr{\mathrm{mr}}               % Most Reactive
% \newcommand\res{\mathrm{res}}                % Resolved
% \newcommand\sgs{\mathrm{sgs}}                % Sub-Grid Scale
% \newcommand\rms{\mathrm{rms}}                % Root-mean-square
% \newcommand\PSR{\inmathrm{PSR}}              % Perfectly Stirred Reactor
% \newcommand\MIL{\inmathrm{MIL}}              % Model Intermittent Lagrangian
% \newcommand\mix{\mathrm{mix}}                % Pure Mixing
% \newcommand\NPR{\inmathit{NPR}}              % Nozzle Pressure Ratio
% \newcommand\IEM{\inmathrm{IEM}}              % Interaction by Exchange with Mean
% \newcommand\traceless{\inmath{d}}            % Deviatrice / Traceless
% \newcommand\seuil{\inmathrm{seuil}}          % Seuil
% \newcommand\eq{\mathrm{eq}}                  % Equilibrium
% \newcommand\diff{\mathrm{diff}}              % Diffusion
% \newcommand\conv{\mathrm{conv}}              % Convection
% \newcommand\igni{\mathrm{igni}}              % Ignition
% \newcommand\propa{\mathrm{propa}}            % Propagation
% \newcommand\unburnt{\inmathrm{u}}            % Unburnt
% \newcommand\burnt{\inmathrm{b}}              % Burnt
% \newcommand\premixed{\mathit{P}}             % Premixed
% \newcommand\npremixed{\mathit{NP}}           % Non-Premixed
% \newcommand\rf{\mathrm{ref}}                 % Reference
% \newcommand\wall{w}                          % Wall
% \newcommand\jet{\mathrm{jet}}                % Jet injection
% \newcommand\SSp{\mathit{SS}}                 % Steady state parameter

% \newcommand\sgsVar{\inmath{\tilde V_\xi}}    % SGS variance
% \newcommand\sqgradzm{\inmath{\vec\nabla\tilde\xi\cdot\vec\nabla\tilde\xi}} % Squared gradient scalar

% % === Amelioration ecriture \tilde{\xi}^2
% %\newcommand\ZtildeS{\inmath{\left.\tilde{\xi}\right.^{2}}} 
% \newcommand\ZtildeS{\inmath{\tilde{\xi\mkern 0mu}^{2}}}
% %\newcommand\ZtildeS{\inmath{\tilde{\xi}{\mathstrut}^{\,2}}}




% % Adimensional numbers
% %%%%%%%%%%%%%%%%%%%%%%%%%%%%%%%%%%%%%%%%%%%%%%%%%%%%%%%%%%%%%%%%%%%%%%%%%%%%%
% \newcommand\adimn[1]{\inmathrm{#1}}   % Font for adimensionalize number

% \newcommand\Rey{\adimn{Re}}  % Reynolds number, cf TeX's \Re real part
% \newcommand\Pran{\adimn{Pr}} % Prandtl number, cf TeX's \Pr product
% \newcommand\Pen{\adimn{Pe}}  % Peclet number
% \newcommand\Sc{\adimn{Sc}}   % Schmidt number
% \newcommand\Str{\adimn{S}}   % Strouhal number
% \newcommand\Le{\adimn{Le}}   % Lewis number
% \newcommand\Ma{\adimn{Ma}}   % Mach number
% \newcommand\Kn{\adimn{Kn}}   % Knudsen number
% \newcommand\Ec{\adimn{Ec}}   % Eckert number
% \newcommand\Fr{\adimn{Fr}}   % Froude number
% \newcommand\Fo{\adimn{Fo}}   % Fourier number
% \newcommand\Ri{\adimn{Ri}}   % Richardson number
% \newcommand\Gra{\adimn{Gr}}  % Grashof number
% \newcommand\Ray{\adimn{Ra}}  % Rayleigh number
% \newcommand\St{\adimn{St}}   % Stanton number
% \newcommand\Nu{\adimn{Nu}}   % Nusselt number
% \newcommand\Da{\adimn{Da}}   % Damköhler number
% \newcommand\Ka{\adimn{Ka}}   % Karlovitz number
% \newcommand\CFL{\adimn{CFL}} % Courant-Friedrichs-Lewy number



% % Cross references
% %%%%%%%%%%%%%%%%%%%%%%%%%%%%%%%%%%%%%%%%%%%%%%%%%%%%%%%%%%%%%%%%%%%%%%%%%%%%%
\usepackage{cleveref}
% % INCOMPATIBLE with the use character ':' and FRENCH in \label 
% % each of the following has two versions
% % \crefname{environmentname}{singular}{plural}, used at mid-sentence
% % \Crefname{environmentname}{singular}{plural}, used at the beginning of a sentence
% \newcommand{\bcref}[1]{[\cref{#1}]}

% \crefname{section}{Sec.}{Secs.}
% %   \Crefname{section}{Section}{Sections}
% \crefname{chapter}{Sec.}{Secs.}
% \crefname{appendix}{Appendix}{Appendices}

% %   \crefname{equation}{}{}
% \crefname{equation}{Eq.}{Eqs.}
% %   \Crefname{equation}{Equation}{Equations}

% \crefname{table}{Table}{Tables}
% %   \Crefname{table}{Tableau}{Tableaux}

% \crefname{figure}{Fig.}{Figs.}
% \Crefname{figure}{Figure}{Figures}

% \crefname{page}{p.}{p.}


% % Raccourcis pour le corpus
% %%%%%%%%%%%%%%%%%%%%%%%%%%%%%%%%%%%%%%%%%%%%%%%%%%%%%%%%%%%%%%%%%%%%%%%%%%%%%
% \newcommand\etal{\mbox{{et al.}}\xspace}
% \newcommand\apriori{\mbox{{a priori}}\xspace}
% \newcommand\aposteriori{\mbox{{a posteriori}}\xspace}
% \newcommand\etc{etc.\xspace}
% \newcommand\eg{e.g.\xspace}
% \newcommand\ie{i.e.\xspace}
% \newcommand\Xth{\textsuperscript{th}\xspace} % for X variable (\th is already taken)

% \newcommand\code[1]{{\fontfamily{pzc}\selectfont #1}\xspace} % Font for code names


% \newcommand\commAM[1]{{\textcolor{blue}{#1}}}   % commentaires
% \newcommand\commAT[1]{\textit{\textcolor{blue}{(#1)}}}   % commentaires
% \newcommand\commGL[1]{\textit{\textcolor{red}{#1}}}     % commentaires
% \newcommand\corrAT[1]{\textcolor{red}{#1}}              % corrections
% \newcommand\corrlatter[1]{\textcolor{red}{#1}}           % a coriger plus tard...

% \setlength\LTleft{0pt} 




% % Coloring
% \usepackage{xcolor}

% \usepackage{txfontsb}
% \let\myBbbk\Bbbk
% \let\Bbbk\relax
% \usepackage[lite]{mtpro2}" dans le preambule du document
% str="chaine de caractere a chercher" ; grep -rni --include=\*.{tex,bib} "$str" .



% Main layouts
% %%%%%%%%%%%%%%%%%%%%%%%%%%%%%%%%%%%%%%%%%%%%%%%%%%%%%%%%%%%%%%%%%%%%%%%%%%%%%
% \usepackage[utf8]{inputenc}
% \usepackage[T1]{fontenc}       % fontes TeX pour la césure des mots et accents PDF
% \usepackage{xspace} 
% \usepackage{textcomp} % symboles divers (autres que mathématiques, ex: degree C) (OBSOLETE 2017 -> ça dépend quelle font est utilisée)
\pdfmapfile{=md-chr7v.map}
\usepackage[super]{nth}
\newcommand{\up}[1]{\textsuperscript{#1}}
\newcommand{\second}{2\up{nd}\xspace}
\newcommand{\seconde}{2\up{nde}\xspace}


% For the PDF document
%%%%%%%%%%%%%%%%%%%%%%%%%%%%%%%%%%%%%%%%%%%%%%%%%%%%%%%%%%%%%%%%%%%%%%%%%%%%%
\usepackage{hyperref}
%\usepackage%[backref=page]
%   {hyperref}
    \hypersetup{
        backref=true,       % permet d'ajouter des liens dans...
        pagebackref=true,   % ...les bibliographies
        hyperindex=true,    % ajoute des liens dans les index.
        colorlinks=true,    % colorise les liens
        breaklinks=true,    % permet le retour a la ligne dans les liens trop longs
        urlcolor= blue,     % couleur des hyperliens
        linkcolor= blue,    % couleur des liens internes
        citecolor=red,      % couleur des citations
    }


% % Comment big section
% %%%%%%%%%%%%%%%%%%%%%%%%%%%%%%%%%%%%%%%%%%%%%%%%%%%%%%%%%%%%%%%%%%%%%%%%%%%%%
% \usepackage{verbatim}
% \usepackage{setspace}


% % Mathematics
% %%%%%%%%%%%%%%%%%%%%%%%%%%%%%%%%%%%%%%%%%%%%%%%%%%%%%%%%%%%%%%%%%%%%%%%%%%%%%
% % \usepackage{amsmath,amssymb,amsfonts,amsbsy} % amsthm
% % \usepackage[version=4]{mhchem}
% % \usepackage{systeme,mathtools}
% \usepackage{amsfonts} % amsthm
\usepackage{xspace}
% \usepackage{amsmath} % amsthm
% % \usepackage[T1]{fontenc}
% \usepackage{lmodern}
% \usepackage{mathtools}% http://ctan.org/pkg/mathtools


\usepackage{graphicx}
\usepackage{amsmath}
\usepackage[version=4]{mhchem}
\usepackage{siunitx}
\usepackage{longtable,tabularx}
% \setlength\LTleft{0pt} 
\usepackage{amsfonts}

% % % Units
% %%%%%%%%%%%%%%%%%%%%%%%%%%%%%%%%%%%%%%%%%%%%%%%%%%%%%%%%%%%%%%%%%%%%%%%%%%%%%
% \usepackage{siunitx}             % unités physiques
% %\usepackage[autolanguage,np]{numprint}
% % séparateur milliers ; symbole des décimales '.' (UK) ou ',' (FR)
% % exemple : \np[N/mm^2]{-123456.1234e3}


% % Tables
% %%%%%%%%%%%%%%%%%%%%%%%%%%%%%%%%%%%%%%%%%%%%%%%%%%%%%%%%%%%%%%%%%%%%%%%%%%%%%
% \usepackage{tabularx}
% \usepackage{booktabs}  
% % \toprule, \midrule, \bottomrule, \addlinespace[<largeur>], ...
% % do not used anymore \hline or |
% % can be used for equation in tables !!!
% \renewcommand{\toprule}{\hline\hline\addlinespace[2pt]}
% \renewcommand{\bottomrule}{\addlinespace[2pt]\hline\hline}
% \usepackage{multirow}  % fusionner lignes d'un tableau
% \usepackage{longtable} % table sur plusieurs pages
% %\usepackage{extra_packages/tabu} % replace tabularx

% \usepackage{threeparttable} % footnote table
% \renewcommand{\tnote}[1]{\textsuperscript{\textcolor{blue}{(\TPTtagStyle{#1})}}}



% % Figures
% %%%%%%%%%%%%%%%%%%%%%%%%%%%%%%%%%%%%%%%%%%%%%%%%%%%%%%%%%%%%%%%%%%%%%%%%%%%%%
% \usepackage{epstopdf}
% \usepackage{rotating} % sideways
% \usepackage{graphicx}
% \usepackage{graphbox} % ex­ten­sion of graph­icx (vertical positionning)
% \usepackage{grffile}             % add file names contain several dots, e.g. "file.name.simulation.pdf"
% \usepackage[percent]{overpic}    % superposer du texte sur image

% %\usepackage{printlen} % print length in human
% %   \uselengthunit{cm}

% \newlength{\mywidth}
% \newlength{\myheight}
% \newlength{\mywidthgraph}
% \newlength{\myheightgraph}
% \newlength{\myheightgraphb}
% \newlength{\mywidthsubfig}

% % === Subfigure ===
% %\usepackage{subfigmat}  % last release 1999 (OBSOLETE...)
% %\usepackage{subfigure}  % last release 2002 (OBSOLETE...)
% %\usepackage{subfig}     % last release 2005 (can be used...)
% \usepackage[
% hypcap=true, 
% %   subrefformat=parens, 
% labelfont=normalfont, % up/it/sl/sc/md/bf(default)/rm/sf/tt/...
% labelformat=simple, 
% ]{subcaption} % last release 2013 INCOMPATIBLE with subfigure or subfig
% \renewcommand\thesubfigure{\bf\alph{subfigure})} % format in caption
% \newcommand{\sublabelsty}[1]{\bf\small{#1)}}


% % Listes, items, descriptions
% %%%%%%%%%%%%%%%%%%%%%%%%%%%%%%%%%%%%%%%%%%%%%%%%%%%%%%%%%%%%%%%%%%%%%%%%%%%%%

% %% IF you use \usepackage[french/frenchb/francais]{babel}, include :
% %% http://daniel.flipo.free.fr/frenchb/frenchb-doc.pdf
% %\frenchbsetup{
% %%   StandardLists=true, % 
% %   StandardItemLabels=true,
% %%   ReduceListSpacing=true,
% %%   CompactItemize=true,
% %   }

% % === PERSONALISER DES LISTES ===
% %\usepackage[olditem,oldenum,defblank,neveradjust]{paralist} % last release 2017
% %\setlength{\plitemsep}{0.5em}
% % http://tug.ctan.org/macros/latex/contrib/paralist/paralist.pdf
% %             ENUMERATE     ITEMIZE       DESCRIPTION
% %   AS PARA   asparaenum    asparaitem    asparadesc
% %   IN PARA   inparaenum    inparaitem    inparadesc
% %   COMPACT   compactenum   compactitem   compactdesc
% % AVOID USING itemize and enumerate environment...
% %
% % === PUIS === (sinon conflit)
% %\usepackage[shortlabels]{enumitem} % last release 2011 (PUT IT AFTER paralist)
% % https://tex.stackexchange.com/questions/176122/problem-with-indent-in-enumeration-of-paralist
% % http://www.edu.upmc.fr/c2i/ressources/latex/aide-memoire.pdf




% %%%%%%%%%%%%%%%%%%%%%%%%%%%%%%%%%%%%%%%%%%%%%%%%%%%%%%%%%%%%%%%%%%%%%%%%%%%%%
% % RACCOURCIS MATHEMATIQUES
% %%%%%%%%%%%%%%%%%%%%%%%%%%%%%%%%%%%%%%%%%%%%%%%%%%%%%%%%%%%%%%%%%%%%%%%%%%%%%

% % === Operateurs AVEC arguments ===
% \newcommand{\dd}{\mathop{}\mathopen{}\mathrm{d}}                  % differentielle d
% \newcommand{\td}[3][]{\frac{\dd^{#1}#2}{\dd{#3}^{#1}}}            % total derivative

% \newcommand{\pd}[3][]{\frac{\partial^{#1}#2}{\partial{#3}^{#1}}}  % partial derivative                            % << NOTE >> : 
% % devrait inverser l'argument #2 et #3 pour pouvoir si besoin juste faire \partial_j{f} = \pd{x_j}{f}
% % désignerait alors juste l'opérateur de dérivée partielle. Idem pour tous les autres...

% \newcommand{\pDelta}[3][]{\dfrac{\delta^{#1}#2}{\delta{#3}^{#1}}} % partial derivative (discrete form)

% \newcommand{\pD}[3][]{\dfrac{D^{#1}#2}{D{#3}^{#1}}}               % particular derivative
% \newcommand{\abs}[1]{\left\lvert #1 \right\rvert}                 % |.| absolute value
% \newcommand{\norme}[1]{\| #1 \|}                                  % ||.|| small norme
% \newcommand{\Norme}[1]{\left\| #1 \right\|}                       % ||.|| large norme
% \newcommand{\lremp}[1]{\left. #1 \right.}                         %  . (empty)
% \newcommand{\lrbkt}[1]{\left( #1 \right)}                         % (.)
% \newcommand{\lrsbkt}[1]{\left[ #1 \right]}                        % [.]
% \newcommand{\lrangle}[1]{\left\langle #1 \right\rangle}           % <.>
% %\newcommand{\blrangle}[2]{#1\langle #2 #1\rangle}                 % <.> big,Big,...
% \newcommand{\lrbrace}[1]{\left\lbrace #1 \right\rbrace}           % {.}
% \newcommand{\blrbrace}[1]{\big\lbrace #1 \big\rbrace}             % {.} big

% %\newcommand{\bs}[1]{\pmb{#1}}                                     % bold symbol
% \newcommand{\bs}[1]{\boldsymbol{#1}}                              % bold symbol (doesn't work with FOURIER font and \nabla)
% %\renewcommand{\vec}{\bs}

% \renewcommand{\tilde}{\widetilde}                                 % ~
% \renewcommand{\hat}{\widehat}                                     % ^
% \renewcommand{\bar}{\overline} 
\newcommand{\pars}[1]{\left(\,{#1}\,\right)}
%                              % -
\usepackage[super]{nth}

% %\newcommand{\range}[4]{\mathopen{}\left#1 #2 \mathpunct{} , #3 \mathclose{}\right#4}
% \newcommand{\range}[4]{\mathopen{}\left#1 #2 ,\: #3 \mathclose{}\right#4}
% \newcommand{\rangeCC}[2]{\range{[}{#1}{#2}{]}}                    % range Close-Close
% \newcommand{\rangeOC}[2]{\range{]}{#1}{#2}{]}}                    % range Open -Close
% \newcommand{\rangeCO}[2]{\range{[}{#1}{#2}{[}}                    % range Close-Open
% \newcommand{\rangeOO}[2]{\range{]}{#1}{#2}{[}}                    % range Open -Open
% \newcommand{\rangeI} [2]{\range{\llbracket}{#1}{#2}{\rrbracket}}  % range Integer

% \newcommand{\colprod}[1]{\textcolor{red}{#1}}                     % COLOR production
% \newcommand{\coldissip}[1]{\textcolor{DeepSkyBlue3}{#1}}          % COLOR dissipation


% % === Operateurs SANS arguments ===
% \DeclareMathOperator\erf{\mathrm{erf}}          % error function
% \DeclareMathOperator\Real{\mathrm{Re}}          % real part
% \DeclareMathOperator\Imag{\mathrm{Im}}          % imaginary part

% \DeclareMathOperator\diverg{\mathrm{div}}       % divergence
% \DeclareMathOperator\rot{\vec{\mathrm{rot}}}    % rotationel
% \DeclareMathOperator\grad{\vec{\mathrm{grad}}}  % gradient

% \DeclareMathOperator{\Lsource}{\mathcal{L}}     % source/sink filtered op
% \DeclareMathOperator{\Lsourced}{\mathit{L}}     % idem for discrete case

% \newcommand\eqdef{\overset{\mathrm{def}}{=}}    % = def
% \newcommand\gdo{\mathcal{O}}                    % grand O
% %\newcommand{\rar}{\rightarrow}                  % ->
% %\newcommand{\rlha}{\rightleftharpoons}          % <=>

% % === LISTE DES VARIABLES, INDICES, EXPOSANTS ===
% % Regle d'uniformisation de syntaxe : 
% %   Nom de la variable = nom en anglais abrégé
% % http://pleasemakeanote.blogspot.fr/2010/07/italics-in-math-equations.html
% % http://folk.uio.no/jornb/howto/latex/jfm2enot.pdf
% \newcommand{\inmath}[1]{\ensuremath{#1}\xspace}
% \newcommand{\inmathrm}[1]{\ensuremath{\mathrm{#1}}\xspace}
% \newcommand{\inmathit}[1]{\ensuremath{\mathit{#1}}\xspace}


% \newcommand\domain{\mathcal{D}}              % Domain
% \newcommand\cst{\inmathrm{cst}}              % Constant
% \newcommand\fuel{\inmathrm{fuel}}            % Fuel stream
% \newcommand\ox{\inmathrm{ox}}                % Oxydizer stream
% \newcommand\fsp{\inmath{F}}                  % Fuel species
% \newcommand\osp{\inmath{O}}                  % Oxydizer species
% \newcommand\psp{\inmath{P}}                  % Product species
% \newcommand\strainrate{\inmath{a}}           % Strain rate
% \newcommand\SDR{\inmath{\chi_\xi}}           % Scalar Dissipation Rate (SDR)
% \newcommand\sgsSDR{\inmath{\chi_\xi^\sgs}}   % SGS Scalar Dissipation Rate
% \newcommand\stc{\mathrm{st}}                 % Stoechiometry
% \newcommand\stoecr{\inmath{r_\stc}}          % Stoechiometric ratio
% \newcommand\mostr{\mathrm{mr}}               % Most Reactive
% \newcommand\res{\mathrm{res}}                % Resolved
% \newcommand\sgs{\mathrm{sgs}}                % Sub-Grid Scale
% \newcommand\rms{\mathrm{rms}}                % Root-mean-square
% \newcommand\PSR{\inmathrm{PSR}}              % Perfectly Stirred Reactor
% \newcommand\MIL{\inmathrm{MIL}}              % Model Intermittent Lagrangian
% \newcommand\mix{\mathrm{mix}}                % Pure Mixing
% \newcommand\NPR{\inmathit{NPR}}              % Nozzle Pressure Ratio
% \newcommand\IEM{\inmathrm{IEM}}              % Interaction by Exchange with Mean
% \newcommand\traceless{\inmath{d}}            % Deviatrice / Traceless
% \newcommand\seuil{\inmathrm{seuil}}          % Seuil
% \newcommand\eq{\mathrm{eq}}                  % Equilibrium
% \newcommand\diff{\mathrm{diff}}              % Diffusion
% \newcommand\conv{\mathrm{conv}}              % Convection
% \newcommand\igni{\mathrm{igni}}              % Ignition
% \newcommand\propa{\mathrm{propa}}            % Propagation
% \newcommand\unburnt{\inmathrm{u}}            % Unburnt
% \newcommand\burnt{\inmathrm{b}}              % Burnt
% \newcommand\premixed{\mathit{P}}             % Premixed
% \newcommand\npremixed{\mathit{NP}}           % Non-Premixed
% \newcommand\rf{\mathrm{ref}}                 % Reference
% \newcommand\wall{w}                          % Wall
% \newcommand\jet{\mathrm{jet}}                % Jet injection
% \newcommand\SSp{\mathit{SS}}                 % Steady state parameter

% \newcommand\sgsVar{\inmath{\tilde V_\xi}}    % SGS variance
% \newcommand\sqgradzm{\inmath{\vec\nabla\tilde\xi\cdot\vec\nabla\tilde\xi}} % Squared gradient scalar

% % === Amelioration ecriture \tilde{\xi}^2
% %\newcommand\ZtildeS{\inmath{\left.\tilde{\xi}\right.^{2}}} 
% \newcommand\ZtildeS{\inmath{\tilde{\xi\mkern 0mu}^{2}}}
% %\newcommand\ZtildeS{\inmath{\tilde{\xi}{\mathstrut}^{\,2}}}




% % Adimensional numbers
% %%%%%%%%%%%%%%%%%%%%%%%%%%%%%%%%%%%%%%%%%%%%%%%%%%%%%%%%%%%%%%%%%%%%%%%%%%%%%
% \newcommand\adimn[1]{\inmathrm{#1}}   % Font for adimensionalize number

% \newcommand\Rey{\adimn{Re}}  % Reynolds number, cf TeX's \Re real part
% \newcommand\Pran{\adimn{Pr}} % Prandtl number, cf TeX's \Pr product
% \newcommand\Pen{\adimn{Pe}}  % Peclet number
% \newcommand\Sc{\adimn{Sc}}   % Schmidt number
% \newcommand\Str{\adimn{S}}   % Strouhal number
% \newcommand\Le{\adimn{Le}}   % Lewis number
% \newcommand\Ma{\adimn{Ma}}   % Mach number
% \newcommand\Kn{\adimn{Kn}}   % Knudsen number
% \newcommand\Ec{\adimn{Ec}}   % Eckert number
% \newcommand\Fr{\adimn{Fr}}   % Froude number
% \newcommand\Fo{\adimn{Fo}}   % Fourier number
% \newcommand\Ri{\adimn{Ri}}   % Richardson number
% \newcommand\Gra{\adimn{Gr}}  % Grashof number
% \newcommand\Ray{\adimn{Ra}}  % Rayleigh number
% \newcommand\St{\adimn{St}}   % Stanton number
% \newcommand\Nu{\adimn{Nu}}   % Nusselt number
% \newcommand\Da{\adimn{Da}}   % Damköhler number
% \newcommand\Ka{\adimn{Ka}}   % Karlovitz number
% \newcommand\CFL{\adimn{CFL}} % Courant-Friedrichs-Lewy number



% % Cross references
% %%%%%%%%%%%%%%%%%%%%%%%%%%%%%%%%%%%%%%%%%%%%%%%%%%%%%%%%%%%%%%%%%%%%%%%%%%%%%
\usepackage{cleveref}
% % INCOMPATIBLE with the use character ':' and FRENCH in \label 
% % each of the following has two versions
% % \crefname{environmentname}{singular}{plural}, used at mid-sentence
% % \Crefname{environmentname}{singular}{plural}, used at the beginning of a sentence
% \newcommand{\bcref}[1]{[\cref{#1}]}

% \crefname{section}{Sec.}{Secs.}
% %   \Crefname{section}{Section}{Sections}
% \crefname{chapter}{Sec.}{Secs.}
% \crefname{appendix}{Appendix}{Appendices}

% %   \crefname{equation}{}{}
% \crefname{equation}{Eq.}{Eqs.}
% %   \Crefname{equation}{Equation}{Equations}

% \crefname{table}{Table}{Tables}
% %   \Crefname{table}{Tableau}{Tableaux}

% \crefname{figure}{Fig.}{Figs.}
% \Crefname{figure}{Figure}{Figures}

% \crefname{page}{p.}{p.}


% % Raccourcis pour le corpus
% %%%%%%%%%%%%%%%%%%%%%%%%%%%%%%%%%%%%%%%%%%%%%%%%%%%%%%%%%%%%%%%%%%%%%%%%%%%%%
% \newcommand\etal{\mbox{{et al.}}\xspace}
% \newcommand\apriori{\mbox{{a priori}}\xspace}
% \newcommand\aposteriori{\mbox{{a posteriori}}\xspace}
% \newcommand\etc{etc.\xspace}
% \newcommand\eg{e.g.\xspace}
% \newcommand\ie{i.e.\xspace}
% \newcommand\Xth{\textsuperscript{th}\xspace} % for X variable (\th is already taken)

% \newcommand\code[1]{{\fontfamily{pzc}\selectfont #1}\xspace} % Font for code names


% \newcommand\commAM[1]{{\textcolor{blue}{#1}}}   % commentaires
% \newcommand\commAT[1]{\textit{\textcolor{blue}{(#1)}}}   % commentaires
% \newcommand\commGL[1]{\textit{\textcolor{red}{#1}}}     % commentaires
% \newcommand\corrAT[1]{\textcolor{red}{#1}}              % corrections
% \newcommand\corrlatter[1]{\textcolor{red}{#1}}           % a coriger plus tard...

% \setlength\LTleft{0pt} 




% % Coloring
% \usepackage{xcolor}

% \usepackage{txfontsb}
% \let\myBbbk\Bbbk
% \let\Bbbk\relax
% \usepackage[lite]{mtpro2}" dans le preambule du document
% str="chaine de caractere a chercher" ; grep -rni --include=\*.{tex,bib} "$str" .



% Main layouts
% %%%%%%%%%%%%%%%%%%%%%%%%%%%%%%%%%%%%%%%%%%%%%%%%%%%%%%%%%%%%%%%%%%%%%%%%%%%%%
% \usepackage[utf8]{inputenc}
% \usepackage[T1]{fontenc}       % fontes TeX pour la césure des mots et accents PDF
% \usepackage{xspace} 
% \usepackage{textcomp} % symboles divers (autres que mathématiques, ex: degree C) (OBSOLETE 2017 -> ça dépend quelle font est utilisée)
\pdfmapfile{=md-chr7v.map}
\usepackage[super]{nth}
\newcommand{\up}[1]{\textsuperscript{#1}}
\newcommand{\second}{2\up{nd}\xspace}
\newcommand{\seconde}{2\up{nde}\xspace}


% For the PDF document
%%%%%%%%%%%%%%%%%%%%%%%%%%%%%%%%%%%%%%%%%%%%%%%%%%%%%%%%%%%%%%%%%%%%%%%%%%%%%
\usepackage{hyperref}
%\usepackage%[backref=page]
%   {hyperref}
    \hypersetup{
        backref=true,       % permet d'ajouter des liens dans...
        pagebackref=true,   % ...les bibliographies
        hyperindex=true,    % ajoute des liens dans les index.
        colorlinks=true,    % colorise les liens
        breaklinks=true,    % permet le retour a la ligne dans les liens trop longs
        urlcolor= blue,     % couleur des hyperliens
        linkcolor= blue,    % couleur des liens internes
        citecolor=red,      % couleur des citations
    }


% % Comment big section
% %%%%%%%%%%%%%%%%%%%%%%%%%%%%%%%%%%%%%%%%%%%%%%%%%%%%%%%%%%%%%%%%%%%%%%%%%%%%%
% \usepackage{verbatim}
% \usepackage{setspace}


% % Mathematics
% %%%%%%%%%%%%%%%%%%%%%%%%%%%%%%%%%%%%%%%%%%%%%%%%%%%%%%%%%%%%%%%%%%%%%%%%%%%%%
% % \usepackage{amsmath,amssymb,amsfonts,amsbsy} % amsthm
% % \usepackage[version=4]{mhchem}
% % \usepackage{systeme,mathtools}
% \usepackage{amsfonts} % amsthm
\usepackage{xspace}
% \usepackage{amsmath} % amsthm
% % \usepackage[T1]{fontenc}
% \usepackage{lmodern}
% \usepackage{mathtools}% http://ctan.org/pkg/mathtools


\usepackage{graphicx}
\usepackage{amsmath}
\usepackage[version=4]{mhchem}
\usepackage{siunitx}
\usepackage{longtable,tabularx}
% \setlength\LTleft{0pt} 
\usepackage{amsfonts}

% % % Units
% %%%%%%%%%%%%%%%%%%%%%%%%%%%%%%%%%%%%%%%%%%%%%%%%%%%%%%%%%%%%%%%%%%%%%%%%%%%%%
% \usepackage{siunitx}             % unités physiques
% %\usepackage[autolanguage,np]{numprint}
% % séparateur milliers ; symbole des décimales '.' (UK) ou ',' (FR)
% % exemple : \np[N/mm^2]{-123456.1234e3}


% % Tables
% %%%%%%%%%%%%%%%%%%%%%%%%%%%%%%%%%%%%%%%%%%%%%%%%%%%%%%%%%%%%%%%%%%%%%%%%%%%%%
% \usepackage{tabularx}
% \usepackage{booktabs}  
% % \toprule, \midrule, \bottomrule, \addlinespace[<largeur>], ...
% % do not used anymore \hline or |
% % can be used for equation in tables !!!
% \renewcommand{\toprule}{\hline\hline\addlinespace[2pt]}
% \renewcommand{\bottomrule}{\addlinespace[2pt]\hline\hline}
% \usepackage{multirow}  % fusionner lignes d'un tableau
% \usepackage{longtable} % table sur plusieurs pages
% %\usepackage{extra_packages/tabu} % replace tabularx

% \usepackage{threeparttable} % footnote table
% \renewcommand{\tnote}[1]{\textsuperscript{\textcolor{blue}{(\TPTtagStyle{#1})}}}



% % Figures
% %%%%%%%%%%%%%%%%%%%%%%%%%%%%%%%%%%%%%%%%%%%%%%%%%%%%%%%%%%%%%%%%%%%%%%%%%%%%%
% \usepackage{epstopdf}
% \usepackage{rotating} % sideways
% \usepackage{graphicx}
% \usepackage{graphbox} % ex­ten­sion of graph­icx (vertical positionning)
% \usepackage{grffile}             % add file names contain several dots, e.g. "file.name.simulation.pdf"
% \usepackage[percent]{overpic}    % superposer du texte sur image

% %\usepackage{printlen} % print length in human
% %   \uselengthunit{cm}

% \newlength{\mywidth}
% \newlength{\myheight}
% \newlength{\mywidthgraph}
% \newlength{\myheightgraph}
% \newlength{\myheightgraphb}
% \newlength{\mywidthsubfig}

% % === Subfigure ===
% %\usepackage{subfigmat}  % last release 1999 (OBSOLETE...)
% %\usepackage{subfigure}  % last release 2002 (OBSOLETE...)
% %\usepackage{subfig}     % last release 2005 (can be used...)
% \usepackage[
% hypcap=true, 
% %   subrefformat=parens, 
% labelfont=normalfont, % up/it/sl/sc/md/bf(default)/rm/sf/tt/...
% labelformat=simple, 
% ]{subcaption} % last release 2013 INCOMPATIBLE with subfigure or subfig
% \renewcommand\thesubfigure{\bf\alph{subfigure})} % format in caption
% \newcommand{\sublabelsty}[1]{\bf\small{#1)}}


% % Listes, items, descriptions
% %%%%%%%%%%%%%%%%%%%%%%%%%%%%%%%%%%%%%%%%%%%%%%%%%%%%%%%%%%%%%%%%%%%%%%%%%%%%%

% %% IF you use \usepackage[french/frenchb/francais]{babel}, include :
% %% http://daniel.flipo.free.fr/frenchb/frenchb-doc.pdf
% %\frenchbsetup{
% %%   StandardLists=true, % 
% %   StandardItemLabels=true,
% %%   ReduceListSpacing=true,
% %%   CompactItemize=true,
% %   }

% % === PERSONALISER DES LISTES ===
% %\usepackage[olditem,oldenum,defblank,neveradjust]{paralist} % last release 2017
% %\setlength{\plitemsep}{0.5em}
% % http://tug.ctan.org/macros/latex/contrib/paralist/paralist.pdf
% %             ENUMERATE     ITEMIZE       DESCRIPTION
% %   AS PARA   asparaenum    asparaitem    asparadesc
% %   IN PARA   inparaenum    inparaitem    inparadesc
% %   COMPACT   compactenum   compactitem   compactdesc
% % AVOID USING itemize and enumerate environment...
% %
% % === PUIS === (sinon conflit)
% %\usepackage[shortlabels]{enumitem} % last release 2011 (PUT IT AFTER paralist)
% % https://tex.stackexchange.com/questions/176122/problem-with-indent-in-enumeration-of-paralist
% % http://www.edu.upmc.fr/c2i/ressources/latex/aide-memoire.pdf




% %%%%%%%%%%%%%%%%%%%%%%%%%%%%%%%%%%%%%%%%%%%%%%%%%%%%%%%%%%%%%%%%%%%%%%%%%%%%%
% % RACCOURCIS MATHEMATIQUES
% %%%%%%%%%%%%%%%%%%%%%%%%%%%%%%%%%%%%%%%%%%%%%%%%%%%%%%%%%%%%%%%%%%%%%%%%%%%%%

% % === Operateurs AVEC arguments ===
% \newcommand{\dd}{\mathop{}\mathopen{}\mathrm{d}}                  % differentielle d
% \newcommand{\td}[3][]{\frac{\dd^{#1}#2}{\dd{#3}^{#1}}}            % total derivative

% \newcommand{\pd}[3][]{\frac{\partial^{#1}#2}{\partial{#3}^{#1}}}  % partial derivative                            % << NOTE >> : 
% % devrait inverser l'argument #2 et #3 pour pouvoir si besoin juste faire \partial_j{f} = \pd{x_j}{f}
% % désignerait alors juste l'opérateur de dérivée partielle. Idem pour tous les autres...

% \newcommand{\pDelta}[3][]{\dfrac{\delta^{#1}#2}{\delta{#3}^{#1}}} % partial derivative (discrete form)

% \newcommand{\pD}[3][]{\dfrac{D^{#1}#2}{D{#3}^{#1}}}               % particular derivative
% \newcommand{\abs}[1]{\left\lvert #1 \right\rvert}                 % |.| absolute value
% \newcommand{\norme}[1]{\| #1 \|}                                  % ||.|| small norme
% \newcommand{\Norme}[1]{\left\| #1 \right\|}                       % ||.|| large norme
% \newcommand{\lremp}[1]{\left. #1 \right.}                         %  . (empty)
% \newcommand{\lrbkt}[1]{\left( #1 \right)}                         % (.)
% \newcommand{\lrsbkt}[1]{\left[ #1 \right]}                        % [.]
% \newcommand{\lrangle}[1]{\left\langle #1 \right\rangle}           % <.>
% %\newcommand{\blrangle}[2]{#1\langle #2 #1\rangle}                 % <.> big,Big,...
% \newcommand{\lrbrace}[1]{\left\lbrace #1 \right\rbrace}           % {.}
% \newcommand{\blrbrace}[1]{\big\lbrace #1 \big\rbrace}             % {.} big

% %\newcommand{\bs}[1]{\pmb{#1}}                                     % bold symbol
% \newcommand{\bs}[1]{\boldsymbol{#1}}                              % bold symbol (doesn't work with FOURIER font and \nabla)
% %\renewcommand{\vec}{\bs}

% \renewcommand{\tilde}{\widetilde}                                 % ~
% \renewcommand{\hat}{\widehat}                                     % ^
% \renewcommand{\bar}{\overline} 
\newcommand{\pars}[1]{\left(\,{#1}\,\right)}
%                              % -
\usepackage[super]{nth}

% %\newcommand{\range}[4]{\mathopen{}\left#1 #2 \mathpunct{} , #3 \mathclose{}\right#4}
% \newcommand{\range}[4]{\mathopen{}\left#1 #2 ,\: #3 \mathclose{}\right#4}
% \newcommand{\rangeCC}[2]{\range{[}{#1}{#2}{]}}                    % range Close-Close
% \newcommand{\rangeOC}[2]{\range{]}{#1}{#2}{]}}                    % range Open -Close
% \newcommand{\rangeCO}[2]{\range{[}{#1}{#2}{[}}                    % range Close-Open
% \newcommand{\rangeOO}[2]{\range{]}{#1}{#2}{[}}                    % range Open -Open
% \newcommand{\rangeI} [2]{\range{\llbracket}{#1}{#2}{\rrbracket}}  % range Integer

% \newcommand{\colprod}[1]{\textcolor{red}{#1}}                     % COLOR production
% \newcommand{\coldissip}[1]{\textcolor{DeepSkyBlue3}{#1}}          % COLOR dissipation


% % === Operateurs SANS arguments ===
% \DeclareMathOperator\erf{\mathrm{erf}}          % error function
% \DeclareMathOperator\Real{\mathrm{Re}}          % real part
% \DeclareMathOperator\Imag{\mathrm{Im}}          % imaginary part

% \DeclareMathOperator\diverg{\mathrm{div}}       % divergence
% \DeclareMathOperator\rot{\vec{\mathrm{rot}}}    % rotationel
% \DeclareMathOperator\grad{\vec{\mathrm{grad}}}  % gradient

% \DeclareMathOperator{\Lsource}{\mathcal{L}}     % source/sink filtered op
% \DeclareMathOperator{\Lsourced}{\mathit{L}}     % idem for discrete case

% \newcommand\eqdef{\overset{\mathrm{def}}{=}}    % = def
% \newcommand\gdo{\mathcal{O}}                    % grand O
% %\newcommand{\rar}{\rightarrow}                  % ->
% %\newcommand{\rlha}{\rightleftharpoons}          % <=>

% % === LISTE DES VARIABLES, INDICES, EXPOSANTS ===
% % Regle d'uniformisation de syntaxe : 
% %   Nom de la variable = nom en anglais abrégé
% % http://pleasemakeanote.blogspot.fr/2010/07/italics-in-math-equations.html
% % http://folk.uio.no/jornb/howto/latex/jfm2enot.pdf
% \newcommand{\inmath}[1]{\ensuremath{#1}\xspace}
% \newcommand{\inmathrm}[1]{\ensuremath{\mathrm{#1}}\xspace}
% \newcommand{\inmathit}[1]{\ensuremath{\mathit{#1}}\xspace}


% \newcommand\domain{\mathcal{D}}              % Domain
% \newcommand\cst{\inmathrm{cst}}              % Constant
% \newcommand\fuel{\inmathrm{fuel}}            % Fuel stream
% \newcommand\ox{\inmathrm{ox}}                % Oxydizer stream
% \newcommand\fsp{\inmath{F}}                  % Fuel species
% \newcommand\osp{\inmath{O}}                  % Oxydizer species
% \newcommand\psp{\inmath{P}}                  % Product species
% \newcommand\strainrate{\inmath{a}}           % Strain rate
% \newcommand\SDR{\inmath{\chi_\xi}}           % Scalar Dissipation Rate (SDR)
% \newcommand\sgsSDR{\inmath{\chi_\xi^\sgs}}   % SGS Scalar Dissipation Rate
% \newcommand\stc{\mathrm{st}}                 % Stoechiometry
% \newcommand\stoecr{\inmath{r_\stc}}          % Stoechiometric ratio
% \newcommand\mostr{\mathrm{mr}}               % Most Reactive
% \newcommand\res{\mathrm{res}}                % Resolved
% \newcommand\sgs{\mathrm{sgs}}                % Sub-Grid Scale
% \newcommand\rms{\mathrm{rms}}                % Root-mean-square
% \newcommand\PSR{\inmathrm{PSR}}              % Perfectly Stirred Reactor
% \newcommand\MIL{\inmathrm{MIL}}              % Model Intermittent Lagrangian
% \newcommand\mix{\mathrm{mix}}                % Pure Mixing
% \newcommand\NPR{\inmathit{NPR}}              % Nozzle Pressure Ratio
% \newcommand\IEM{\inmathrm{IEM}}              % Interaction by Exchange with Mean
% \newcommand\traceless{\inmath{d}}            % Deviatrice / Traceless
% \newcommand\seuil{\inmathrm{seuil}}          % Seuil
% \newcommand\eq{\mathrm{eq}}                  % Equilibrium
% \newcommand\diff{\mathrm{diff}}              % Diffusion
% \newcommand\conv{\mathrm{conv}}              % Convection
% \newcommand\igni{\mathrm{igni}}              % Ignition
% \newcommand\propa{\mathrm{propa}}            % Propagation
% \newcommand\unburnt{\inmathrm{u}}            % Unburnt
% \newcommand\burnt{\inmathrm{b}}              % Burnt
% \newcommand\premixed{\mathit{P}}             % Premixed
% \newcommand\npremixed{\mathit{NP}}           % Non-Premixed
% \newcommand\rf{\mathrm{ref}}                 % Reference
% \newcommand\wall{w}                          % Wall
% \newcommand\jet{\mathrm{jet}}                % Jet injection
% \newcommand\SSp{\mathit{SS}}                 % Steady state parameter

% \newcommand\sgsVar{\inmath{\tilde V_\xi}}    % SGS variance
% \newcommand\sqgradzm{\inmath{\vec\nabla\tilde\xi\cdot\vec\nabla\tilde\xi}} % Squared gradient scalar

% % === Amelioration ecriture \tilde{\xi}^2
% %\newcommand\ZtildeS{\inmath{\left.\tilde{\xi}\right.^{2}}} 
% \newcommand\ZtildeS{\inmath{\tilde{\xi\mkern 0mu}^{2}}}
% %\newcommand\ZtildeS{\inmath{\tilde{\xi}{\mathstrut}^{\,2}}}




% % Adimensional numbers
% %%%%%%%%%%%%%%%%%%%%%%%%%%%%%%%%%%%%%%%%%%%%%%%%%%%%%%%%%%%%%%%%%%%%%%%%%%%%%
% \newcommand\adimn[1]{\inmathrm{#1}}   % Font for adimensionalize number

% \newcommand\Rey{\adimn{Re}}  % Reynolds number, cf TeX's \Re real part
% \newcommand\Pran{\adimn{Pr}} % Prandtl number, cf TeX's \Pr product
% \newcommand\Pen{\adimn{Pe}}  % Peclet number
% \newcommand\Sc{\adimn{Sc}}   % Schmidt number
% \newcommand\Str{\adimn{S}}   % Strouhal number
% \newcommand\Le{\adimn{Le}}   % Lewis number
% \newcommand\Ma{\adimn{Ma}}   % Mach number
% \newcommand\Kn{\adimn{Kn}}   % Knudsen number
% \newcommand\Ec{\adimn{Ec}}   % Eckert number
% \newcommand\Fr{\adimn{Fr}}   % Froude number
% \newcommand\Fo{\adimn{Fo}}   % Fourier number
% \newcommand\Ri{\adimn{Ri}}   % Richardson number
% \newcommand\Gra{\adimn{Gr}}  % Grashof number
% \newcommand\Ray{\adimn{Ra}}  % Rayleigh number
% \newcommand\St{\adimn{St}}   % Stanton number
% \newcommand\Nu{\adimn{Nu}}   % Nusselt number
% \newcommand\Da{\adimn{Da}}   % Damköhler number
% \newcommand\Ka{\adimn{Ka}}   % Karlovitz number
% \newcommand\CFL{\adimn{CFL}} % Courant-Friedrichs-Lewy number



% % Cross references
% %%%%%%%%%%%%%%%%%%%%%%%%%%%%%%%%%%%%%%%%%%%%%%%%%%%%%%%%%%%%%%%%%%%%%%%%%%%%%
\usepackage{cleveref}
% % INCOMPATIBLE with the use character ':' and FRENCH in \label 
% % each of the following has two versions
% % \crefname{environmentname}{singular}{plural}, used at mid-sentence
% % \Crefname{environmentname}{singular}{plural}, used at the beginning of a sentence
% \newcommand{\bcref}[1]{[\cref{#1}]}

% \crefname{section}{Sec.}{Secs.}
% %   \Crefname{section}{Section}{Sections}
% \crefname{chapter}{Sec.}{Secs.}
% \crefname{appendix}{Appendix}{Appendices}

% %   \crefname{equation}{}{}
% \crefname{equation}{Eq.}{Eqs.}
% %   \Crefname{equation}{Equation}{Equations}

% \crefname{table}{Table}{Tables}
% %   \Crefname{table}{Tableau}{Tableaux}

% \crefname{figure}{Fig.}{Figs.}
% \Crefname{figure}{Figure}{Figures}

% \crefname{page}{p.}{p.}


% % Raccourcis pour le corpus
% %%%%%%%%%%%%%%%%%%%%%%%%%%%%%%%%%%%%%%%%%%%%%%%%%%%%%%%%%%%%%%%%%%%%%%%%%%%%%
% \newcommand\etal{\mbox{{et al.}}\xspace}
% \newcommand\apriori{\mbox{{a priori}}\xspace}
% \newcommand\aposteriori{\mbox{{a posteriori}}\xspace}
% \newcommand\etc{etc.\xspace}
% \newcommand\eg{e.g.\xspace}
% \newcommand\ie{i.e.\xspace}
% \newcommand\Xth{\textsuperscript{th}\xspace} % for X variable (\th is already taken)

% \newcommand\code[1]{{\fontfamily{pzc}\selectfont #1}\xspace} % Font for code names


% \newcommand\commAM[1]{{\textcolor{blue}{#1}}}   % commentaires
% \newcommand\commAT[1]{\textit{\textcolor{blue}{(#1)}}}   % commentaires
% \newcommand\commGL[1]{\textit{\textcolor{red}{#1}}}     % commentaires
% \newcommand\corrAT[1]{\textcolor{red}{#1}}              % corrections
% \newcommand\corrlatter[1]{\textcolor{red}{#1}}           % a coriger plus tard...

% \setlength\LTleft{0pt} 




% % Coloring
% \usepackage{xcolor}

% \usepackage{txfontsb}
% \let\myBbbk\Bbbk
% \let\Bbbk\relax
% \usepackage[lite]{mtpro2}